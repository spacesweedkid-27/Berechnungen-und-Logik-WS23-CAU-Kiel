\documentclass[12pt, a4paper]{article}

\usepackage[ngerman]{babel} 
\usepackage[T1]{fontenc}
\usepackage{amsfonts} 
\usepackage{setspace}
\usepackage{amsmath}
\usepackage{amssymb}
\usepackage{titling}


\newcommand*{\qed}{\null\nobreak\hfill\ensuremath{\square}}
\newcommand*{\puffer}{\text{ }\text{ }\text{ }\text{ }}
\newcommand*{\gedanke}{\textbf{-- }}
\newcommand*{\gap}{\text{ }}


\pagestyle{plain}
\allowdisplaybreaks

\setlength{\droptitle}{-11em}
\setlength{\jot}{12pt}

\title{Berechnungen und Logik\\Hausaufgabenserie 1}
\author{Henri Heyden, Nike Pulow \\ \small stu240825, stu239549}
\date{}


\begin{document}
\maketitle

\doublespacing
\subsection*{A1}
Definiere folgende Mengen: \(odd := \{2a+1 | a \in \mathbb Z\}\), \(even := \{2a | a \in \mathbb Z\}\)\\
sei \(n \in \mathbb Z\), man sieht leicht, dass \(\{odd, even\}\) Partition über \(\mathbb Z\) ist.
\subsubsection*{a)}
Wir werden zeigen, dass für ein \(n\in odd\) gilt, dann \(n^2\in odd\) gilt. \\
Es gilt per Definition: \(\forall n \in odd: \exists a \in \mathbb Z: 2a+1 = n\). \\
Für dieses \(a\) gilt dann \(n^2 = (2a+1)^2 = 4a^2 + 4a + 1\). \\
Da \(a \in \mathbb Z\) gilt: \(4a^2 \in even \wedge 4a \in even \wedge 1 \in odd\), somit ist \(4a^2 + 4a + 1\) ungerade, also \(n^2\) ungerade. \qed
\subsubsection*{b)}
Wir werden zeigen, dass für ein \(n^2\in odd\) gilt, dann \(n\in odd\) gilt. \\
Hierfür zeigen wir die Kontraposition, also \(n \not\in odd \Longrightarrow n^2 \not\in odd\). \\
Nach Voraussetzung ist die äquivalent zu: \(n \in even \Longrightarrow n^2 \in even\). \\
Es gilt per Definition: \(\forall n \in even: \exists a \in \mathbb Z: 2a = n\). \\
Für dieses \(a\) gilt dann \(n^2 = (2a)^2 = 4a^2 \in even\). Damit ist \(n^2\) gerade und die Kontraposition ist gezeigt \qed
\subsection*{A2}
\subsubsection*{a)}
Gilt \(\mathbb N := \{1, 2, ...\}\) die Menge der natürlichen Zahlen ab 1, dann gilt: \[\mathbb N : \begin{cases}
    1 \in \mathbb{N} \\
    n \in \mathbb{N} \Rightarrow n+1 \in \mathbb{N}
\end{cases}\]
\subsubsection*{b)}
Vorausgesetzt, $\mathbb{N}$ ist wie oben definiert:\\
Basiselemente: $aab$\\
Induktionsregel: Es gilt $w=a^{n+1}b^{n}$ für alle $w \in L \subset \Sigma ^{*} $ mit $n \in \mathbb{N}$.
\subsection*{A3}
\subsubsection*{a)}
3,6,9,12
\subsubsection*{b)}
IA: Bekannt ist, dass $3 \in M$ gilt. Wir wissen also, dass für $n = 1$ $3n \in M$ gilt, da $3 \cdot 1 = 3 \in M$.\\
IV: Sei $M'$ eine Menge, welche nach mehrfacher Anwendung der induktiven Regel erzeugt wurde und es gelte $m = 3n$ für alle $n \in \mathbb{N}$ und $m \in M'$.\\
IS: Betrachte nun: $3(n+1) = 3n + 3 \cdot 1 = 3n + 3$.\\
Nach IA ist bekannt, dass $3 \in M$ gilt und nach IV gilt $3n \in M'$.\\
\qed
\subsection*{A4}
$\{ ab, aaa, c \} \subset L_1$ \puffer \puffer \puffer \puffer \puffer \puffer \gap $L_1$ ist endlich.\\
$\{ aa, bb, cc \} \subset L_2$ \puffer \puffer \puffer \puffer \puffer \puffer \gap \gap $L_2$ ist unendlich.\\
$\{ ababab, aaaaaaaaa, ccc\} \subset L_3$ \puffer \puffer $L_3$ ist unendlich.\\
$L_4 = \{ \}$ \puffer \puffer \puffer \puffer \puffer \puffer \puffer \puffer \puffer \gap $L_4$ ist endlich.\\
$\{ aba, aaaa, ca \} \subset L_5$ \puffer \puffer \puffer \puffer \puffer $L_5$ ist unendlich.\\
$\{ aa, bb, cc \} \subset L_6$ \puffer \puffer \puffer \puffer \puffer \puffer \gap $L_6$ ist unendlich.\\
\subsection*{A5}
Man sieht leicht, dass die Aussage für endliche Mengen gilt\\
(es gilt \(|M| < 2^{|M|} = |\mathcal P(M)|\)). \\ \\ Per Induktion lässt sich die Aussage auch für abzählbar unendliche Mengen beweisen:\\
Induktionsbasis: \(|\emptyset| = 0 < 1 = 2^{|\emptyset|} = |\mathcal P(\emptyset)|\). \\
Induktionsschritt: Sei also angenommen für eine Menge \(M\), dass \(|M| < \mathcal{P}(M)\) gilt. Dann gilt für ein Element \(x \not\in M\):\\
\(|M \cup \{x\}| = |M| + 1 < 2\cdot |\mathcal P (M)| \le |\mathcal P(M \cup \{x\})|\). Dies folgt aus der Überlegung, dass in der Potenzmenge von \(M \cup \{x\}\) mindestens alle Teilmengen von \(M\) vorkommen müssen. Des Weiteren müssen in dieser Menge auch alle Teilmengen von \(M\) liegen, welche noch dazu ein \(x\) bekommen, da diese Mengen dann Teilmengen von \(M \cup \{x\}\) sind. \\ \\
Nun für alle Mengen im allgemeinen, \gedanke also auch überabzählbare Mengen: \\
Sei \(M\) Menge. Wir werden zeigen, dass keine Bijektion \(f: M \rightarrow \mathcal P (M)\) existiert, woraus \(|M| \ne |\mathcal P (M)|\) folgt. \\
Nehme an, dass \(f\) wäre bijektiv. \\
Definiere \(\Omega := \{ x \in M | x \not\in f(x)\}.\) Dann ist \(\Omega\) Teilmenge von \(M\), also Element von \(\mathcal P (M)\). Da \(f\) bijektiv ist, gilt: \(\Omega \in \text{img}(f)\), und damit\footnote[1]{Hier ist wichtig zu beachten, dass bei einer Bijektion immer genau ein Element der Urbildmenge eines Elements existiert, also ein Inverses}: \(f^\leftarrow (\Omega) \in M\). Dieses Inverse nennen wir \(\omega\). Für dieses gilt dann natürlich: \(f(\omega) = \Omega\). \\
Jedoch gilt: \(\omega \not\in f(\omega)\) also \(\omega \not\in \Omega\), nach Definition von \(\Omega\). Die Annahme der Bijektivität führt somit zu dem Widerspruch \(\omega \in \Omega \wedge \omega \not\in \Omega\). \\
Nun um den Fall \(|M| > |\mathcal P (M)|\) auszuschließen,\\
betrachte \(f: M \rightarrow \mathcal P(M), x \mapsto \{x\}\). \(f\) ist injektiv aber nicht surjektiv.\\
Somit gilt: \(|M| \le |\mathcal P (M)|\) insgesamt \(|M| < |\mathcal P (M)|\). \qed
\end{document}
