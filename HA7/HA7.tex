\documentclass[12pt, a4paper]{article}

\usepackage[ngerman]{babel} 
\usepackage[T1]{fontenc}
\usepackage{amsfonts} 
\usepackage{setspace}
\usepackage{amsmath}
\usepackage{amssymb}
\usepackage{titling}
\usepackage{csquotes} % for \textquote{}
\usepackage{hyperref}
\usepackage{tikz}
\usetikzlibrary{arrows, automata, positioning}

\newcommand*{\qed}{\null\nobreak\hfill\ensuremath{\square}}
\newcommand*{\lqed}{\null\nobreak\hfill\ensuremath{\blacksquare}}
\newcommand*{\puffer}{\text{ }\text{ }\text{ }\text{ }}
\newcommand*{\gedanke}{\textbf{-- }}
\newcommand*{\gap}{\text{ }}
\newcommand*{\setDef}{\gap|\gap}
\newcommand*{\vor}{\textbf{Vor.:} \gap}
\newcommand*{\beh}{\textbf{Beh.:} \gap}
\newcommand*{\bew}{\textbf{Bew.:} \gap}
% Hab länger gebraucht um zu realisieren, dass das ne gute Idee wäre
\newcommand*{\R}{\mathbb R}

\newenvironment{noalign*}
 {\setlength{\abovedisplayskip}{0pt}\setlength{\belowdisplayskip}{0pt}%
  \csname flalign*\endcsname}
 {\csname endflalign*\endcsname\ignorespacesafterend}



\pagestyle{plain}
\allowdisplaybreaks

\setlength{\droptitle}{-11em}
\setlength{\jot}{12pt}
%\setlength{\hoffset}{-1in}     Wenn nötig
%\setlength{\textwidth}{535pt}  Wenn nötig

\title{Berechnungen und Logik\\Hausaufgabenserie 8}
\author{Henri Heyden, Nike Pulow \\ \small stu240825, stu239549}
\date{}


\begin{document}
\maketitle

\onehalfspacing
\vspace*{-2cm}
\subsection*{A1}
\subsubsection*{a)}
\subsection*{A2}
\vor \(L_1, L_2\) sind Turing erkennbar, sodass \(L_1 \cup L_2\) und \(L_1 \cap L_2\) Turing entscheidbar. \\
\beh \(L_1\) ist Turing entscheidbar. \\
\bew Da \(L_1 \cup L_2\) und \(L_1 \cap L_2\) Turing entscheidbar sind existieren die Turingmaschinen \(M_\cap\) und \(M_\cup\), sodass \(M_\cap\) die Sprache \(L_1 \cap L_2\) akzeptiert oder sonst verwirft und \(\overline M_\cup\) die Sprache \(\overline{L_1 \cup L_2}\) akzeptiert oder sonst verwirft. \\
Da \(L_1\) Turing erkennbar ist, existiert eine Turingmaschine \(\overline M_1\), die genau \(\overline L_1\) akzeptiert. \\
Nun definieren wir die Turingmaschine \(M\) mit folgender Verhaltensweise für ein Wort \(w\):\\
akzeptiert \(\overline M_\cup\) \(w\), dann verwerfe \(w\), \\
sonst: akzeptiert \(M_\cap\) \(w\), dann akzeptiere \(w\), \\
sonst: verwirft \(\overline M_1\) \(w\), dann verwerfe \(w\),
sonst: akzeptiere \(w\). \\
Dann ist \(M\) eine Turingmaschine, die \(L_1\) entscheidet, somit ist \(L_1\) entscheidbar \qed
\subsection*{A3}
\beh Eine Turingmaschine mit einem beidseitig unendlichem Band entscheidet genau die gleichen Sprachen, wie eine einseitig unendliche Turingmaschine. \\
\bew Wir teilen den Beweis in zwei Richtungen auf: \\
\textbf{1. } \textquote{\(\Rightarrow\)} \\
Es wurde in der Vorlesung bereits gezeigt, dass k-bändige Turingmaschine gleich mächtig sind, wie einbändige, deswegen zeigen wir, dass für jede beidseitige Turingmaschine eine beidbandige Turingmaschine existiert. \\
Betrachten wir eine beidseitige Turingmaschine. \\
Dann existiert folgende beidbandige Turingmaschine: Wir nennen Band 1 \textquote{rechtes Band} und Band 2 \textquote{linkes Band}. Für jegliches beschriftetes Band, dass verwendet wird, werden das linke Band und das rechte Band so aufgeteilt, dass das linke Band die gespiegelte linke Hälfte des originalen Bandes ist und das rechte Band die rechte Hälfte des originalen Bandes ist. Die Mitte ist dabei der Start des rechten Bandes.\\
Transformiere nun beide Bänder nochmal, so, dass ein Buchstabe, der nicht im Bandalphabet rechts bzw. links vom linken bzw. rechtem Band angefügt wird. Dieser Buchstabe repräsentiert die Mitte des originalen Bandes. \\
Nun adjustieren wir die Transitionsfunktion so, dass folgendes gilt:\\
Starte auf dem zweiten Buchstaben des rechten Bandes\footnote[1]{wir betrachten hier eine zweibandige Turingmaschiene, wessen Bänder nicht synchronisiert sind, also wo wir nicht sozusagen Tupel betrachten, dass diese Turingmaschinen mit den synchronen k-bandigen äquivalent sind in der Stärke, ließe sich leicht zeigen, aber wir werden nicht uns nicht darauf fokussieren in diesem Beweis, um ihn nicht unnötig in die Länge zu ziehen, \gedanke wie sonst.}.\\
Wir rechnen so lange genau so, wie in der ursprünglichen Turingmaschine, bis wir (falls überhaupt) die Markierung erreichen. \\
Dann rechne mit invertierten Bewegungsaktionen bei der zweiten Stelle des linken Bandes weiter (also nicht auf der Markierung, sonst würden wir ja wieder zurückspringen).\\
Wenn wieder die Markierung erreicht wird, fahre wieder im \textquote{originalen} Modus rechts weiter auf der zweiten Stelle. \\
Wir ersetzen sozusagen die Linke unendliche Hälfte der ursprünglichen Turingmaschine mit einem einzelnen Band. \\
Eine andere Möglichkeit mit einer Turingmaschine mit einem Band das Problem zu lösen ist es eine zweiseitige Turingmaschine auf der einbandigen zu simulieren, wenn man annimmt, dass sich jede mathematische Struktur über eine Turingmaschine simulieren lässt.\\
\textbf{2. } \textquote{\(\Leftarrow\)} \\
Betrachten wir eine einseitige Turingmaschine, dann existiert eine Turingmaschine mit zweiseitigem Band, welche genau die gleichen Transitionsfunktionen besitzt. \\
Es ist uns schließlich egal, wie viel \textquote{Platz} links von dem Ursprung ist, da wir nie diesen überschreiten werden. \\
\\
Somit sind beide Richtungen gezeigt. \qed
\end{document}