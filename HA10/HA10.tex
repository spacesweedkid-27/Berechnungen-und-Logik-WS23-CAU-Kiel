\documentclass[12pt, a4paper]{article}

\usepackage[ngerman]{babel} 
\usepackage[T1]{fontenc}
\usepackage{amsfonts} 
\usepackage{setspace}
\usepackage{amsmath}
\usepackage{amssymb}
\usepackage{titling}
\usepackage{csquotes} % for \textquote{}
\usepackage{hyperref}
\usepackage{tikz}
\usepackage{stix}
\usepackage{stmaryrd} % \llbracket and \rrbracket
\usetikzlibrary{arrows, automata, positioning}

\newcommand*{\qed}{\null\nobreak\hfill\ensuremath{\square}}
\newcommand*{\lqed}{\null\nobreak\hfill\ensuremath{\blacksquare}}
\newcommand*{\puffer}{\text{ }\text{ }\text{ }\text{ }}
\newcommand*{\gedanke}{\textbf{-- }}
\newcommand*{\gap}{\text{ }}
\newcommand*{\setDef}{\gap|\gap}
\newcommand*{\vor}{\textbf{Vor.:} \gap}
\newcommand*{\beh}{\textbf{Beh.:} \gap}
\newcommand*{\bew}{\textbf{Bew.:} \gap}
% Hab länger gebraucht um zu realisieren, dass das ne gute Idee wäre
\newcommand*{\R}{\mathbb R}
% Das hier wird sehr nett werden
\newcommand*{\vDashR}{\mathrel{\reflectbox{\(\vDash\)}}}
\newcommand*{\LeftrightvDash}{\vDash\vDashR}
\newcommand{\lb}{\llbracket}
\newcommand{\rb}{\rrbracket}
\newcommand{\rbb}{\rrbracket_{\beta}}

\newenvironment{noalign*}
 {\setlength{\abovedisplayskip}{0pt}\setlength{\belowdisplayskip}{0pt}%
  \csname flalign*\endcsname}
 {\csname endflalign*\endcsname\ignorespacesafterend}



\pagestyle{plain}
\allowdisplaybreaks

\setlength{\droptitle}{-11em}
\setlength{\jot}{12pt}
%\setlength{\hoffset}{-1in}     Wenn nötig
%\setlength{\textwidth}{535pt}  Wenn nötig

\title{Berechnungen und Logik\\Hausaufgabenserie 10}
\author{Henri Heyden, Nike Pulow \\ \small stu240825, stu239549}
\date{}


\begin{document}
\maketitle

\singlespacing
\vspace*{-2cm}
\subsection*{HA 1}
Vorberechnung: \\
Um beide Beweise etwas kürzer zu machen, führen wir eine Resolution durch von \(\Phi\), da wir in a) und b) genau die gleichen Schritte nochmal sonst machen würden nur mit noch eingefügten Formeln.\\
\begin{noalign*}
    & 1 & \{X_0\} & & \text{Voraussetzung} \\
    & 2 & \{\neg X_0, X_1\} & & \text{Voraussetzung} \\
    & 3 & \{X_1\} & & \text{Resolution mit \(X_0\) aus 1 und 2} \\
    & 4 & \{\neg X_1, X_3\} & & \text{Voraussetzung} \\
    & 5 & \{X_3\} & & \text{Resolution mit \(X_1\) aus 3 und 4} \\
    & 6 & \{\neg X_2, X_4\} & & \text{Voraussetzung} \\
    & 7 & \{\neg X_4, X_3\} & & \text{Voraussetzung} \\
    & 8 & \{\neg X_2, X_3\} & & \text{Resolution mit \(X_4\) aus 6 und 7} \\
    & 9 & \{\neg X_4, X_2\} & & \text{Voraussetzung} \\
    & 10 & \{\neg X_4, X_3\} & & \text{Resolution mit \(X_2\) aus 8 und 9} \\
\end{noalign*}
Es bleibt nur noch die Klauselmenge \(\{\{X_3\}, \{\neg X_4, X_3\}\}\). \\
Beobachte: \(\bigwedge\bigvee\{\{X_3\}, \{\neg X_4, X_3\}\} \LeftrightvDash \bigwedge\bigvee\{\{X_3\}\}\) aufgrund der Konjunktion.
\subsubsection*{a)}
Wir wenden alle Resolutionsschritte und die Umformung zuletzt auf \(\Phi \cup \{\neg X_3\}\) an, und erhalten \(\bigwedge\bigvee\{\{X_3\}, \{\neg X_3\}\} \vDash \bot\).\\
Nach Lemma ist \(\Phi \vDash X_3\) gezeigt. \qed
\subsubsection*{b)}
Wir wenden alle Resolutionsschritte und die Umformung zuletzt auf \(\Phi \cup \{\neg X_4\}\) an, und erhalten \(\bigwedge\bigvee\{\{X_3\}, \{\neg X_4\}\}\) also \(X_3 \wedge X_4\).\\
Da keine Resolutionsschritte mehr anwendbar sind\\
und offenbar nicht \(X_3 \wedge \neg X_4 \vDash \bot\) gilt (Siehe Belegung \(\beta := \{X_3 \mapsto 1, X_4 \mapsto 0\}\)), \\
können wir nicht schlussfolgern \(\Phi \vDash X_4\), jedoch da das Resolutionskalkül vollständig ist, können wir \(\Phi \vDash X_4\) ausschließen. \qed
\subsection*{HA 2}
\subsubsection*{a)}
\(t_1\) ist \(\mathcal S\)-Term, hier werden alle Formalismen für die Signatur \(\mathcal S\) eingehalten. \\
\(t_2\) ist nicht \(\mathcal S\)-Term, wir haben \(\doteq\) so definiert, dass links und rechts \(\mathcal S\)-Terme stehen, was auch stimmt, jedoch ergibt dieser entstehende Baum eine \(\mathcal S\)-Formel, kein \(\mathcal S\)-Term. \\
\(t_3\) ist nicht \(\mathcal S\)-Term, wir schreiben Variablen hier klein und nicht groß, also ist \(X_0\) nicht Unter\(\mathcal S\)-term. \\
\(t_4\) ist nicht \(\mathcal S\)-Term, da \(T\) Relation ist, Relationen dürfen nicht in \(\mathcal S\)-Termen vorkommen.
\subsubsection*{b)}
\(\varphi_1\) ist nicht \(\mathcal S\)-Formel, da \(T(c)\) Formel ist und nicht \(\mathcal S\)-Term. \\
\(\varphi_2\) ist nicht \(\mathcal S\)-Formel, da rechts von \textquote{\(\doteq\)} kein \(\mathcal S\)-Term steht. \\
\(\varphi_3\) ist nicht \(\mathcal S\)-Formel, da die Variable \(x_0\) nicht \(\mathcal S\)-Formel ist. \\
\(\varphi_4\) ist nicht \(\mathcal S\)-Formel, da links von dem Junktor \textquote{\(\wedge\)} ein \(\mathcal S\)-Term steht und nicht eine \(\mathcal S\)-Formel.
\subsection*{HA 3}
\subsubsection*{i = 0)}
Definiere für die Struktur \(\mathcal A_0\) das Universum \(A_0 := \{\text{in}, \text{out}, 1, 2, 3\}\). \\
Definiere \(E := \{(\text{in}, 2), (2, 3), (3, \text{out}), (3,1), (1,2)\}\). \\
Dann gilt \(E \subseteq A_0^2\), des Weiteren hat jede Konstante aus \(\mathcal S\) ein Komplement in \(A_0\).\\
Somit ist \(\mathcal A_0\) \(\mathcal S\)-Struktur und \(\mathcal A_0\) modelliert den Graph \(G_0\).
\subsubsection*{i = 1)}
Definiere für die Struktur \(\mathcal A_1\) das Universum \(A_1 := \{\text{in}, \text{out}, 1, 2, 3, 4, 5\}\). \\
Definiere \(E := \{(\text{in}, 2), (2, 3), (3, 4), (4,2), (4,5), (5,4), (1,2), (1,5), (5, \text{out})\}\). \\
Dann gilt \(E \subseteq A_1^2\), des Weiteren hat jede Konstante aus \(\mathcal S\) ein Komplement in \(A_1\).\\
Somit ist \(\mathcal A_1\) \(\mathcal S\)-Struktur und \(\mathcal A_1\) modelliert den Graph \(G_1\).
\subsubsection*{i = 2)}
Definiere für die Struktur \(\mathcal A_2\) das Universum \(A_2 := \{\text{in}, \text{out}, \text{...}, 1, 2, 3\}\). \\
Definiere \(E := \{(\text{in}, 1), (1, \text{out}), (1, 2), (2,3), (3, \text{...}), (1,1), (2,2), (3,3)\}\). \\
Dann gilt \(E \subseteq A_2^2\), des Weiteren hat jede Konstante aus \(\mathcal S\) ein Komplement in \(A_2\).\\
Somit ist \(\mathcal A_2\) \(\mathcal S\)-Struktur und \(\mathcal A_2\) modelliert den Graph \(G_2\).
\subsection*{HA 4}
\subsubsection*{i = 1)}
Für \(\lb \varphi \rbb^{\mathcal{A}_{1,0}} = 0\) definiere \(\mathcal{A}_{1,0}\) wie folgt:\\
Das Universum \(A = \mathbb{N}_0\), die Konstante \(c^{\mathcal{A}_{1,0}} = 0\), die Funktion \(f^{\mathcal{A}_{1,0}}(x) = 
x \cdot c\), für alle \(x \in A\), sowie die Relation \(R^{\mathcal{A}_{1,0}} = \text{ } <\).\\\\
Für \(\lb \varphi \rbb^{\mathcal{A}_{1,1}} = 1\) definiere \(\mathcal{A}_{1,1}\) wie folgt:\\
Das Universum \(A = \mathbb{N}_{0}\), die Konstante \(c^{\mathcal{A}_{1,1}} = 1\), die Funktion \(f^{\mathcal{A}_{1,1}}(x) = 
x+c\), für alle \(x \in A\) sowie die Relation \(R^{\mathcal{A}_{1,1}} = \text{ } <\).
\subsubsection*{i = 2)}
Für \(\lb \varphi \rbb^{\mathcal{A}_{2,0}} = 0\) definiere \(\mathcal{A}_{2,0}\) wie folgt:\\
Das Universum \(A = \{1,2\}\), die Konstante \(c^{\mathcal{A}_{2,0}} = 1\), die Funktion \(f^{\mathcal{A}_{2,0}}(x) = 1\), für alle \(x \in A\),
sowie die Relation \(R^{\mathcal{A}_{2,0}} = \text{ } =\).\\\\
Für \(\lb \varphi \rbb^{\mathcal{A}_{2,1}} = 1\) definiere \(\mathcal{A}_{2,1}\) wie folgt:\\
Das Universum \(A = \{1\}\), die Konstante \(c^{\mathcal{A}_{2,1}} = 0\), die Funktion \(f^{\mathcal{A}_{2,1}}(x) = 1\), für alle \(x \in A\),
sowie die Relation \(R^{\mathcal{A}_{2,1}} = \text{ } =\).
\end{document}