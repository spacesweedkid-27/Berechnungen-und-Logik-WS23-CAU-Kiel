\documentclass[12pt, a4paper]{article}

\usepackage[ngerman]{babel} 
\usepackage[T1]{fontenc}
\usepackage{amsfonts} 
\usepackage{setspace}
\usepackage{amsmath}
\usepackage{amssymb}
\usepackage{titling}
\usepackage{csquotes} % for \textquote{}
\usepackage{tikz}
\usetikzlibrary{arrows, automata, positioning}

\newcommand*{\qed}{\null\nobreak\hfill\ensuremath{\square}}
\newcommand*{\puffer}{\text{ }\text{ }\text{ }\text{ }}
\newcommand*{\gedanke}{\textbf{-- }}
\newcommand*{\gap}{\text{ }}
\newcommand*{\setDef}{\gap|\gap}
\newcommand*{\vor}{\textbf{Vor.:} \gap}
\newcommand*{\beh}{\textbf{Beh.:} \gap}
\newcommand*{\bew}{\textbf{Bew.:} \gap}
% Hab länger gebraucht um zu realisieren, dass das ne gute Idee wäre
\newcommand*{\R}{\mathbb R}


\pagestyle{plain}
\allowdisplaybreaks

\setlength{\droptitle}{-11em}
\setlength{\jot}{12pt}
\setlength{\hoffset}{-0.5in}
\setlength{\textwidth}{435pt}

\title{Berechnungen und Logik\\Hausaufgabenserie 6}
\author{Henri Heyden, Nike Pulow \\ \small stu240825, stu239549}
\date{}


\begin{document}
\maketitle

\doublespacing
\vspace*{-2cm}
\subsection*{A1}
\subsubsection*{a)}
\vor Sei \(A\) Alphabet. \(L_1 := \{w \in \{a,b\}^* \setDef w = w^R\}\), wobei \(\cdot^R : A^* \rightarrow A^*\) wie folgt:\\
IB für \(w \in A^*, a \in A\): \((a \cdot r)^R := r^R \cdot a\)\\
IS für \(a \in A \cup \{\epsilon\}\): \(a^R := a\) \\
Außerdem: \(\cdot//\cdot: \mathbb{N}^2 \rightarrow \mathbb N, (a,b) \mapsto \left\lfloor \frac{a}{b} \right\rfloor\) und: \(\cdot\%\cdot : \mathbb{N}^2 \rightarrow \mathbb N, (a,b) \mapsto a - b \cdot \left\lfloor \frac{a}{b} \right\rfloor\).\\
Diese beiden Operatoren sind Implementierungen der bekannten \\ abgerundeten ganzzahligen Division und dem Modulo. \\
\beh \(L_1\) ist nicht regulär. \\
\bew Wir zeigen dies, indem wir die Kontraposition des Pumping-Lemmas zeigen,\\ d.h. es ist zu zeigen:
\(\forall m \in \mathbb N_{\ge 1}: \exists w \in \{a,b\}^m: \forall x,y,z \in \{a,b\}^*, w = xyz, |y| \ge 1:\)\\
\(\puffer\puffer\puffer\puffer\puffer\puffer\puffer\exists i,j \in \mathbb{N}, v \in \{a,b\}^*: xy^izv \in L_1 \not\Leftrightarrow xy^jzv \in L_1\)\\
Sei also \(m \in \mathbb N_{\ge 1}\).\\
Wähle \(w = a^{(m//2) - 1} \cdot b \cdot b^{m\% 2} \cdot b \cdot a^{(m//2) - 1}\).\\
Im Falle, dass \(m\) ungerade ist, haben wir also ein \textquote{extra} \(b\) in der Mitte, sodass das Wort \(w\) Palindrom ist und gleichzeitig immer noch wahrlich genauso lang wie \(m\) ist. \\
Sei dementsprechend \(xyz\) eine Zerlegung von \(w\) mit \(|y| \ge 1\). Wir wählen \(v = (xyz)^R\). \\
Dann gilt offenbar für \(i = 1\): \(xy^izv = xyzv = xyz(xyz)^R \in L_1\). \\
Jedoch gilt für \(j = 2\): \(xy^jzv = xyyzv = xyyz(xyz)^R \not\in L_1\), da \(|y| \ge 1\) vorausgesetzt. \\
Also gilt \(xy^izv \in L_1 \not\Rightarrow xy^jzv \in L_1\), woraus folgt: \(xy^izv \in L_1 \not\Leftrightarrow xy^jzv \in L_1\),\\
\gedanke was zu zeigen war. \qed
\pagebreak
\subsubsection*{b)}
\(L_2\) ist die Menge der Palindrome über \(\{a,b\}\) vereint mit der Menge an Wörtern, die keine Palindrome sind, es folgt: \(L_2 = \{a,b\}^*\), somit lässt sich ein DEA konstruieren, der einfach jedes Wort aus \(\{a,b\}^*\) akzeptiert:\vspace*{-0.5cm}
\[\mathcal A := \{\{\text{OK}\}, \{a,b\}, \{((\text{OK}, a), \text{OK}), ((\text{OK}, b), \text{OK})\}, \text{OK}, \{\text{OK}\}\}\]
Diese Funktionsschreibweise ist etwas umständlich, aber bei kleinen Relationen lässt sich die Tupelschreibweise verwenden. Man sieht, dass diese Transitionsfunktion immer OK auf sich selbst abbildet und wohldefiniert ist.\\
Es ergibt sich folgendes Zustandsdiagramm:

\vspace*{-1.8cm}
\hspace*{+10cm}
\begin{tikzpicture}[->, >=stealth', node distance=2cm, semithick, block/.style={maximum size =1.5cm},  font=\tiny, align = center]
    \node[initial, state, accepting] (ok) {\text{OK}};
    \path (ok) edge [loop above] node {a} (ok)
          (ok) edge [loop below] node {b} (ok);
\end{tikzpicture} \vspace*{-1cm}
\onehalfspacing
\subsubsection*{c)}
\vor \(L_2 := \{0^{(2^n)} \setDef n \in \mathbb N_0\}\). \\
\beh \(L_2\) ist nicht regulär. \\
\bew Wir verwenden das gleiche Schema, wie in \textbf{a)}.\\Die Quantorenkette werden wir jetzt nicht ausschreiben. \\
Sei also \(m\in \mathbb{N}_{\ge 1}\), wähle \(w := 0^{(2^n)}\) und \(xyz := w\) mit \(|y|\ge 1\). \\
Wähle \(v := xyz\), dann gilt: \(xy^1zv = xyzxyz \in L_2\), also existiert ein \(i = 1\). \\
Sei \(j = 2\), dann gilt folgendes: \(xy^jzv = xyyzv = yyxzv\). Es gilt, dass jeder Buchstabe der Wörter 0 ist, weswegen wir uns nur die Kardinalitäten dieser Kompositionen anschauen werden.\\
Es gilt: \(|y| \in [1, 2^m]_\mathbb N\), da \(|xyz| = 2^m\) gilt, was die maximale Kardinalität von \(y\) ist, und da \(|y| \ge 1\) gilt. \\
Es folgt: \(|yyxzv| = |y| + 2^m + 2^m = |y| + 2^{m+1}\). \\
Damit diese Kardinalität eine Zweierpotenz sein soll, müsste \(|y| \ge 2^{m+1}\) gelten, aber da \(|y|\) maximal \(2^m\) ist, fehlt uns im besten Fall immer noch ein Viertel.\\
Damit gilt somit \(xy^izv \in L_2 \not\Rightarrow xy^jzv \in L_2\), \gedanke was wie zuvor, zu zeigen war. \qed
\subsubsection*{d)}
\end{document}