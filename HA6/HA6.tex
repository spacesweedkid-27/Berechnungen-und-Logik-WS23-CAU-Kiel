\documentclass[12pt, a4paper]{article}

\usepackage[ngerman]{babel} 
\usepackage[T1]{fontenc}
\usepackage{amsfonts} 
\usepackage{setspace}
\usepackage{amsmath}
\usepackage{amssymb}
\usepackage{titling}
\usepackage{csquotes} % for \textquote{}
\usepackage{hyperref}
\usepackage{tikz}
\usetikzlibrary{arrows, automata, positioning}

\newcommand*{\qed}{\null\nobreak\hfill\ensuremath{\square}}
\newcommand*{\lqed}{\null\nobreak\hfill\ensuremath{\blacksquare}}
\newcommand*{\puffer}{\text{ }\text{ }\text{ }\text{ }}
\newcommand*{\gedanke}{\textbf{-- }}
\newcommand*{\gap}{\text{ }}
\newcommand*{\setDef}{\gap|\gap}
\newcommand*{\vor}{\textbf{Vor.:} \gap}
\newcommand*{\beh}{\textbf{Beh.:} \gap}
\newcommand*{\bew}{\textbf{Bew.:} \gap}
% Hab länger gebraucht um zu realisieren, dass das ne gute Idee wäre
\newcommand*{\R}{\mathbb R}

\newenvironment{noalign*}
 {\setlength{\abovedisplayskip}{0pt}\setlength{\belowdisplayskip}{0pt}%
  \csname flalign*\endcsname}
 {\csname endflalign*\endcsname\ignorespacesafterend}


\pagestyle{plain}
\allowdisplaybreaks

\setlength{\droptitle}{-11em}
\setlength{\jot}{12pt}
\setlength{\hoffset}{-0.5in}
\setlength{\textwidth}{435pt}

\title{Berechnungen und Logik\\Hausaufgabenserie 6}
\author{Henri Heyden, Nike Pulow \\ \small stu240825, stu239549}
\date{}


\begin{document}
\maketitle

\doublespacing
\vspace*{-2cm}
\subsection*{A1}
\subsubsection*{a)}
\vor Sei \(A\) Alphabet. \(L_1 := \{w \in \{a,b\}^* \setDef w = w^R\}\), wobei \(\cdot^R : A^* \rightarrow A^*\) wie folgt:\\
IB für \(w \in A^*, a \in A\): \((a \cdot r)^R := r^R \cdot a\)\\
IS für \(a \in A \cup \{\epsilon\}\): \(a^R := a\) \\
Außerdem: \(\cdot//\cdot: \mathbb{N}^2 \rightarrow \mathbb N, (a,b) \mapsto \left\lfloor \frac{a}{b} \right\rfloor\) und: \(\cdot\%\cdot : \mathbb{N}^2 \rightarrow \mathbb N, (a,b) \mapsto a - b \cdot \left\lfloor \frac{a}{b} \right\rfloor\).\\
Diese beiden Operatoren sind Implementierungen der bekannten \\ abgerundeten ganzzahligen Division und dem Modulo. \\
\beh \(L_1\) ist nicht regulär. \\
\bew Wir zeigen dies, indem wir die Kontraposition des Pumping-Lemmas zeigen,\\ d.h. es ist zu zeigen:
\(\forall m \in \mathbb N_{\ge 1}: \exists w \in \{a,b\}^m: \forall x,y,z \in \{a,b\}^*, w = xyz, |y| \ge 1:\)\\
\(\puffer\puffer\puffer\puffer\puffer\puffer\puffer\exists i,j \in \mathbb{N}, v \in \{a,b\}^*: xy^izv \in L_1 \not\Leftrightarrow xy^jzv \in L_1\)\\
Sei also \(m \in \mathbb N_{\ge 1}\).\\
Wähle \(w = a^{(m//2) - 1} \cdot b \cdot b^{m\% 2} \cdot b \cdot a^{(m//2) - 1}\).\\
Im Falle, dass \(m\) ungerade ist, haben wir also ein \textquote{extra} \(b\) in der Mitte, sodass das Wort \(w\) Palindrom ist und gleichzeitig immer noch wahrlich genauso lang wie \(m\) ist. \\
Sei dementsprechend \(xyz\) eine Zerlegung von \(w\) mit \(|y| \ge 1\). Wir wählen \(v = (xyz)^R\). \\
Dann gilt offenbar für \(i = 1\): \(xy^izv = xyzv = xyz(xyz)^R \in L_1\). \\
Jedoch gilt für \(j = 2\): \(xy^jzv = xyyzv = xyyz(xyz)^R \not\in L_1\), da \(|y| \ge 1\) vorausgesetzt. \\
Also gilt \(xy^izv \in L_1 \not\Rightarrow xy^jzv \in L_1\), woraus folgt: \(xy^izv \in L_1 \not\Leftrightarrow xy^jzv \in L_1\),\\
\gedanke was zu zeigen war. \qed
\pagebreak
\subsubsection*{b)}
\(L_2\) ist die Menge der Palindrome über \(\{a,b\}\) vereint mit der Menge an Wörtern, die keine Palindrome sind, es folgt: \(L_2 = \{a,b\}^*\), somit lässt sich ein DEA konstruieren, der einfach jedes Wort aus \(\{a,b\}^*\) akzeptiert:\vspace*{-0.5cm}
\[\mathcal A := \{\{\text{OK}\}, \{a,b\}, \{((\text{OK}, a), \text{OK}), ((\text{OK}, b), \text{OK})\}, \text{OK}, \{\text{OK}\}\}\]
Diese Funktionsschreibweise ist etwas umständlich, aber bei kleinen Relationen lässt sich die Tupelschreibweise verwenden. Man sieht, dass diese Transitionsfunktion immer OK auf sich selbst abbildet und wohldefiniert ist.\\
Es ergibt sich folgendes Zustandsdiagramm:

\vspace*{-1.8cm}
\hspace*{+10cm}
\begin{tikzpicture}[->, >=stealth', node distance=2cm, semithick, block/.style={maximum size =1.5cm},  font=\tiny, align = center]
    \node[initial, state, accepting] (ok) {\text{OK}};
    \path (ok) edge [loop above] node {a} (ok)
          (ok) edge [loop below] node {b} (ok);
\end{tikzpicture} \vspace*{-1cm}
\onehalfspacing
\subsubsection*{c)}
\vor \(L_3 := \{0^{(2^n)} \setDef n \in \mathbb N_0\}\). \\
\beh \(L_3\) ist nicht regulär. \\
\bew Wir verwenden das gleiche Schema, wie in \textbf{a)}.\\Die Quantorenkette werden wir jetzt nicht ausschreiben. \\
Sei also \(m\in \mathbb{N}_{\ge 1}\), wähle \(w := 0^{(2^n)}\) und \(xyz := w\) mit \(|y|\ge 1\). \\
Wähle \(v := xyz\), dann gilt: \(xy^1zv = xyzxyz \in L_3\), also existiert ein \(i = 1\). \\
Sei \(j = 2\), dann gilt folgendes: \(xy^jzv = xyyzv = yyxzv\). Es gilt, dass jeder Buchstabe der Wörter 0 ist, weswegen wir uns nur die Kardinalitäten dieser Kompositionen anschauen werden.\\
Es gilt: \(|y| \in [1, 2^m]_\mathbb N\), da \(|xyz| = 2^m\) gilt, was die maximale Kardinalität von \(y\) ist, und da \(|y| \ge 1\) gilt. \\
Es folgt: \(|yyxzv| = |y| + 2^m + 2^m = |y| + 2^{m+1}\). \\
Damit diese Kardinalität eine Zweierpotenz sein soll, müsste \(|y| \ge 2^{m+1}\) gelten, aber da \(|y|\) maximal \(2^m\) ist, fehlt uns im besten Fall immer noch ein Viertel.\\
Damit gilt somit \(xy^izv \in L_3 \not\Rightarrow xy^jzv \in L_3\), \gedanke was wie zuvor, zu zeigen war. \qed
\subsubsection*{d)}
\vor \(L_4 := \{w^Rbwbw^R \in \{a,b\}^* \setDef w \in \{b\}^*\}\) \\
\beh \(L_4\) ist regulär. \\
\bew Es gilt \(\forall w \in \{b\}^*: w^R = w\), da \(w\) schließlich nur aus dem gleichen Buchstaben besteht. Betrachte die folgende Umformung:
\begin{noalign*}
    & \{w^Rbwbw^R \in \{a,b\}^* \setDef w \in \{b\}^*\} & \text{|Folgerung, Komposition ist immer aus \(\{b\}^*\)} \\
    = & \{wbwbw \setDef w \in \{b\}^*\} & \text{| Reihenfolge macht keinen Unterschied, weil nur \(b\)'s im Wort sind} \\
    = & \{bbwww \setDef w \in \{b\}^*\} & \text{| Rausziehen (wir betrachten das Bild)} \\
    = & bb \cdot \{www \setDef w \in \{b\}^*\} & \text{| Def. \(\cdot^*\) für Mengen} \\
    = & bb \cdot \{w \in \{b\}^* \setDef |w| \in 3\mathbb N\}
\end{noalign*}
Wir suchen also Wörter für die Folgendes gilt: \(w : w \in \{b\}^* |w| = 3n + 2\) für \(n \in \mathbb N\). \\
Hiermit lässt sich ein DEA komposieren, der \(L_4\) erkennt, welcher dargestellt ist in folgender Illustration:
\begin{center} \vspace*{-0.7cm}
\begin{tikzpicture}[->, >=stealth', node distance=2cm, semithick, block/.style={maximum size =1.5cm},  font=\small, align = center]
    \node[initial, state] (e) {\(\epsilon\)};
    \node[state, right of = e] (1) {1};
    \node[state, accepting, right of = 1] (ack) {\text{OK}};
    \node[state, right = 0.8cm, above right of = ack] (+1) {+1};
    \node[state, right = 0.8cm, below right of = ack] (+2) {+2};
    \path   (e) edge [above] node {b} (1)
            (1) edge [above] node {b} (ack)
            (ack) edge [above left] node {b} (+1)
            (+1) edge [right] node {b} (+2)
            (+2) edge [below left] node {b} (ack);
\end{tikzpicture}
\end{center}
Dieser Automat erkennt genau \(L_4\), so ist \(L_4\) regulär. \qed \pagebreak
\subsection*{A2}
\doublespacing
\subsubsection*{a)}
\vor \(A := \{\uparrow, \rightarrow, \downarrow, \leftarrow\}, L_1 := \{p \in A^* \setDef p \text{ beginnt und endet am selben Punkt.}\}\) \\
\beh \(L_1\) ist nicht regulär. \\
\bew Wir zeigen wieder nach Pumping-Lemma, d.h. es ist zu zeigen: \\
\puffer\puffer\puffer\(\forall m \in \mathbb N_{\ge 1}:\) \\
\puffer\puffer\puffer\(\exists w \in A^m: \forall x,y,z \in A^*, w = xyz, |y| \ge 1:\)\\
\puffer\puffer\puffer\(\exists i,j \in \mathbb{N}, v \in A^*: xy^izv \in L_1 \not\Leftrightarrow xy^jzv \in L_1\)\\
Sei also \(m \in \mathbb N_{\ge 1}\). Setze \(w := \rightarrow^m\) und \(v := \leftarrow^m\).\footnote[1]{\textquote{\textit{Perfectly balanced, \gedanke as all things should be.}}} \\
Sei \(xyz := w\) mit \(|y| \ge 1\) Zerlegung von \(w\). \\
Dann gilt \textbf{Fall 1 (\(i = 1\)):} \(xy^izv = xyzv \in L_1\), da \(|w| = |v|\). \\
Des Weiteren gilt \textbf{Fall 2 (\(j = 2\)):} \(xy^jzv = xyyzv = yxyzv = ywz = \rightarrow^{|y|} \cdot \rightarrow^{|m|} \cdot \leftarrow^{|m|}\) \\
Da \(|y| \ge 1\) folgt: \(\rightarrow^{|y|} \ne \epsilon\). Mit \(y \in \{\rightarrow\}^+\) gilt somit \(xy^jzw \not\in L_1\). \\
Also gilt: \(\exists i,j \in \mathbb{N}, v \in A^*: xy^izv \in L_1 \not\Leftrightarrow xy^jzv \in L_1\), \gedanke was zu zeigen war. \qed
\pagebreak
\subsubsection*{b)}
\vor \(L_2 := \{a^{m \cdot n} \setDef n,m \in \mathbb N_{> 1}\} \cup \{\epsilon, a\}\),\gap\gap \(\mathbb P \subset \mathbb N: 2 \in \mathbb P, p \in \mathbb P \Rightarrow \forall n \in \mathbb N_{> 1}: p\cdot n \not\in \mathbb P\) \\
\beh \(L_2\) ist nicht regulär. \\
\bew Es gilt: \(L_2 = \{a^{m \cdot n} \setDef n,m \in \mathbb N_{> 1}\} \cup \{\epsilon, a\} = \{a^p \setDef p \not\in \mathbb P\}\). Sei hierfür bewusst, dass die Menge der Primzahlen gleiche Kardinalität zur Menge der natürlichen Zahlen hat, also dass \(|\mathbb P| = |\mathbb N|\) gilt. \\
\textbf{Lemma} \(\forall L: L \in \text{Reg} \Leftrightarrow \overline{L} \in \text{Reg}\): \\
\textquote{\(\Rightarrow\)}:\\
Sei \(M := \{Q, A, \delta, I, F\}\) NEA der \(L\) akzeptiert, dann ist \(\overline{M} := \{Q, A, \delta, F, I\}\) ein NEA, der \(\overline{L}\) akzeptiert. Folgt aus Definition von NEA und Akzeptanz bzw. Berechnung.\\
\textquote{\(\Leftarrow\)}:\\
folgt aus vorherigen Teil, denn \(F,I\) lassen sich zurücktauschen. \lqed\\
Nach dem Lemma zeigen wir somit, dass \(\overline{L_2}\) nicht regulär ist. Es gilt: \(\overline{L_2} = \{a^p \setDef p \in \mathbb P\}\)\\
Wir zeigen wieder mittels des Pumping-Lemmas.\\
Nehme also \(m \in \mathbb N\), sei \(w := a^m\), sei \(xyz := w\) Zerlegung von \(w\) mit \(|y| \ge 1\). \\
Sei auch \(v := a^{p - |m|}\), wobei \(p := \min(\mathbb P_{\ge m})\) gilt.\footnote[2]{Es \textquote{sei} hier auch wieder bewusst, dass es abzählbar unendliche Primzahlen gibt.}\\
Dann gilt Folgendes: \\
Sei \(i := 0\), dann gilt: \(xy^izv = xzv = a^{m-1} \cdot a^{p-m} = a^{m-1+p-m} = a^{p-1} \not\in \overline{L_2}\), da nach Def. \(\mathbb P\) der Abstand von zwei Primzahlen größer als 2 sein muss, denn \(2 | (p\pm 1)\) gilt.\\
Sei \(j := 0\), dann gilt: \(xy^jzv = xyzv = a^m \gap\gap \cdot a^{p-m} = a^{m+p-m} \gap\gap\gap = a^{p} \in \overline{L_2}\), nach \(\overline{L_2}\).\\
Dann gilt: \(\exists i,j \in \mathbb{N}, v \in \{a\}^*: xy^izv \in \overline{L_2} \not\Leftrightarrow xy^jzv \in \overline{L_2}\). Damit ist \(\overline{L_2}\) nicht regulär, folglich ist auch \(L_2\) nicht regulär, \gedanke was zu zeigen war. \qed

\subsection*{A3}
\subsubsection*{a)}
Im Folgenden definieren wir \(DTM_{Pal}\) als die deterministische Turing Maschine, welche die Sprache \(L = \{ww^R \in A^* \mid w \in A^*\}\)
für ein Alphabet \(A = \{a,b\}\) entscheidet.\\
Die Idee ist, dass immer der erste noch nicht gelesene Buchstabe auf dem Band mit dem 
letzten noch nicht gelesenen Buchstaben auf dem Band vergleichen wird. Sind diese beiden 
Buchstaben gleich, fährt die Turing Maschine mit dem nächsten Buchstaben-Paar fort. Es ist also 
essentiell, dass es aus dem ersten Zustand \(q_0\) zwei Zustände \(q_1\) und \(q_2\) gibt, in 
denen gespeichert wird, welches Symbol gelesen wird. In diesen Zuständen wird dann geblieben, bis 
das Ende des Bandes erreicht wird (also ein \textvisiblespace gelesen wird). Dann entscheidet die 
Turing Maschine, ob das am Ende des Bandes gelesene Zeichen gleich dem Zeichen am Anfang des Bandes 
ist. Ist dem so, fährt der Kopf zurück an den Anfang des Bandes - es muss also jedes 
Zeichen, welches Teil eines Vergleiches ist, durch ein \textvisiblespace ersetzt werden. 
Jene Zeichen hingegen, welche nur gelesen werden, wenn der Kopf von einem Ende des Bandes zum 
anderen fährt, werden nicht verändert. \\
Ist nun ein Zeichen am Ende des Bandes nicht das gleiche, wie jenes, welches zum Vergleich am 
Anfang des Bandes gelesen wurde, so wird der verwerfende Zustand \(q_{rej}\) erreicht. Es wird \(q_{rej}\) auch 
erreicht, wenn nur ein zu vergleichendes Zeichen übrig ist, nicht aber ein zweites. In diesem 
Fall ist die Anzahl der Buchstaben des Eingabewortes ungerade und gehört damit nicht zu \(L\). Liest die 
Turing Maschine im Zustand \(q_0\) jedoch ein \textvisiblespace, dann ist das Eingabewort entweder 
leer, oder das Eingabewort ein Palindrom gerader Länge und damit Teil der zu entscheidenen Sprache \(L\). 
Dann wird der Zustand \(q_{acc}\) erreicht. \\
Wir definieren nun die Maschine \(DTM_{Pal}\) formal und als Diagramm: \footnote[3]{Beim nochmaligen 
Lesen der Aufgabenstellung stellen wir fest, dass lediglich ein Zustandsdiagramm von Nöten gewesen 
wäre. Wir sehen die längere Ausführung und folgende Definition hier nun als Übung an, welche gleichzeitig auch 
deutlich machen sollte, warum die Korrektheit unserer Konstruktion gilt, wenngleich nicht 
explizit ein Nebensatz dahingehend formuliert wurde.}\\
\(Q = \{q_0, q_1, q_2, q_3, q_4, q_5, q_{rej}, q_{acc}\}\) mit \(q_0\) als initialen Zustand, 
\(q_{rej}\) als verwerfenden Zustand und \(q_{acc}\) als akzeptierenden Zustand.\\
\(A = \{a,b\}\)\\
\(\Gamma = \{a,b,\) \textvisiblespace\(\}\)\\
\begin{tabular}{c c c c}
    \(\delta\) & \((q, a)\) & \(\rightarrow\) & \((q, a, r)\)\\
    \hline
            &\((q_0, a)\)                       && \((q_1\), \textvisiblespace, \(R)\)\\
            &\((q_0, b)\)                       && \((q_2\), \textvisiblespace, \(R)\)\\
            &\((q_0\), \textvisiblespace\()\)   && \((q_{acc}\)\textvisiblespace, \(R)\)\\
            &\((q_1, a)\)                       && \((q_1, a, R)\)\\
            &\((q_1, b)\)                       && \((q_1, b, R)\)\\
            &\((q_1\), \textvisiblespace \()\)  && \((q_3\), \textvisiblespace, \(L)\)\\
            &\((q_2, a)\)                       && \((q_2, a, R)\)\\
            &\((q_2, b)\)                       && \((q_2, b, R)\)\\
            &\((q_2\), \textvisiblespace \()\)  && \((q_5\), \textvisiblespace, \(L)\)\\
            &\((q_3, a)\)                       && \((q_4\), \textvisiblespace, \(L)\)\\
            &\((q_3, b)\)                       && \((q_{rej}\), \textvisiblespace, \(L)\)\\
            &\((q_3\), \textvisiblespace \()\)  && \((q_{rej}\), \textvisiblespace, \(L)\)\\
            &\((q_4, a)\)                       && \((q_4, a, R)\)\\
            &\((q_4, b)\)                       && \((q_4, b, R)\)\\
            &\((q_4\), \textvisiblespace \()\)  && \((q_0\), \textvisiblespace, \(L)\)\\
            &\((q_5, a)\)                       && \((q_{rej}\), \textvisiblespace, \(L)\)\\
            &\((q_5, b)\)                       && \((q_4\), \textvisiblespace, \(L)\)\\
            &\((q_5\), \textvisiblespace \()\)  && \((q_{rej}\), \textvisiblespace, \(L)\)\\
\end{tabular}
\begin{center}
    \begin{tikzpicture}[->, >=stealth', node distance=3cm, semithick,  font=\small, align = center, sloped]
    
        \node[initial,state](0)[right=0cm]{\(q_0\)};
        \node[state](1)[right=3cm]{\(q_1\)};
        \node[state](2)[below of = 1]{\(q_2\)};
        \node[state](3)[right=6cm]{\(q_3\)};
        \node[state](4)[below of = 3]{\(q_4\)};
        \node[state](5)[below of = 4]{\(q_5\)};
        \node[state,accepting](rej)[right = 9cm]{rej};
        \node[state,accepting](acc)[below of = 0]{acc};

        \path   (0) edge [above] node {\(a \rightarrow\) \textvisiblespace, \(R\)} (1)
                (0) edge [above] node {\(b \rightarrow\) \textvisiblespace, \(R\)} (2)
                (0) edge [below] node {\textvisiblespace \(\rightarrow\) \textvisiblespace, \(R\)} (acc)
                (1) edge [loop above] node {\(a \rightarrow a, R\)\\\(b \rightarrow b, R\)} (1)
                (1) edge [above] node {\textvisiblespace \(\rightarrow\) \textvisiblespace, \(L\)} (3)
                (2) edge [loop below] node {\(a \rightarrow a, R\)\\\(b \rightarrow b, R\)} (2)
                (2) edge [above] node {\textvisiblespace \(\rightarrow\) \textvisiblespace, \(L\)} (5)
                (3) edge [above] node {\textvisiblespace \(\rightarrow\)\textvisiblespace, \(L\)\\\(b \rightarrow\) \textvisiblespace, \(L\)} (rej)
                (3) edge [above] node {\(a \rightarrow\) \textvisiblespace, \(L\)} (4)
                (4) edge [loop right, above] node {\(a \rightarrow a, R\)\\\(b \rightarrow b, R\)} (4)
                (4) edge [above] node {\textvisiblespace \(\rightarrow\) \textvisiblespace, \(L\)} (0)
                (5) edge [below] node {\(b \rightarrow\) \textvisiblespace, \(L\)} (4)
                (5) edge [below, bend right = 40] node {\(a \rightarrow\) \textvisiblespace, \(L\)\\\textvisiblespace \(\rightarrow\) \textvisiblespace, \(L\)} (rej);
    \end{tikzpicture}
\end{center}
\subsubsection*{b)}
Für das Wort \(abba\) erhalten wir mit \(DTM_{Pal}\) folgende Berechnung: \\
1. \(q_0 b b a\) \textvisiblespace\\
2. \textvisiblespace \(q_1 b a\) \textvisiblespace\\
3. \textvisiblespace \(b q_1 a\) \textvisiblespace\\
4. \textvisiblespace \(b b q_1\) \textvisiblespace\\
5. \textvisiblespace \( b b a q_3\)\\
6. \textvisiblespace \( b b q_4\) \textvisiblespace\\
7. \textvisiblespace \(b q_4\) \textvisiblespace \textvisiblespace\\
8. \textvisiblespace \(q_4 b\) \textvisiblespace \textvisiblespace\\
9. \(q_0 b b\) \textvisiblespace \textvisiblespace\\
10. \textvisiblespace \(q_2 b\) \textvisiblespace \textvisiblespace\\
11. \textvisiblespace \textvisiblespace \(q_2\) \textvisiblespace \textvisiblespace\\
12. \textvisiblespace \textvisiblespace \( b q_5\) \textvisiblespace\\
13. \textvisiblespace \textvisiblespace \(q_4\) \textvisiblespace \textvisiblespace\\
14. \textvisiblespace \(q_0\) \textvisiblespace \textvisiblespace \textvisiblespace\\
15. \textvisiblespace \textvisiblespace \(q_{acc}\) \textvisiblespace \textvisiblespace\\
\subsubsection*{c)}
Für das Wort \(abbb\) erhalten wir mit \(DTM_{Pal}\) folgende Berechnung: \\
1. \(q_0 b b b\) \textvisiblespace\\
2. \textvisiblespace \(q_1 b b \) \textvisiblespace\\
3. \textvisiblespace \(b q_1 b \) \textvisiblespace\\
4. \textvisiblespace \(b b q_1 \) \textvisiblespace\\
5. \textvisiblespace \(b b b q_3\)\\
6. \textvisiblespace \(b b q_{rej}\) \textvisiblespace\\

\subsection*{A4}
\subsubsection*{a)}
Diese Aussage ist korrekt. Es lässt sich für jede Sprache ein DEA konstruieren, von 
dessen Startzustand aus für alle Eingaben ein weiterer Zustand \({q_1}_M\) existiert, 
welcher bei Bedarf eingehende Transitionen besitzt. 

\subsubsection*{b)}
Diese Aussage ist inkorrekt. Es kann für Wörter beliebig langer Länge kein DEA konzipiert 
werden, dessen akzeptierenden Zustände keine ausgehenden Transitionen akzeptiert. \\
Betrachte beispielsweise einen Automaten \(M\), welcher alle Wörter einer Sprache \(A=\{a,b\}\) 
erkennt, welche mit dem Buchstaben \(a\) beginnen. Dieser lässt sich für beliebig lange Wörter 
nur mit einer ausgehenden Transition für den akzeptierenden Zustand konstruieren.
\begin{center}
    \begin{tikzpicture}[->, >=stealth', node distance=2cm, semithick,  font=\small, align = center, sloped]
    
        \node[initial,state](0)[right=0cm]{\(q_0\)};
        \node[state](2)[below of = 0]{\(q_2\)};
        \node[state,accepting](1)[right=2cm]{\(q_1\)};

        \path   (0) edge [above] node {a} (1)
                (0) edge [above] node {b} (2)
                (2) edge [loop below] node {a,b} (2)
                (1) edge [loop above] node {a,b} (1);
    
    \end{tikzpicture}
\end{center}

\subsubsection*{c)}
Diese Aussage ist inkorrekt. Betrachte dafür wieder den Automaten aus Aufgabenteil b. 
Es ist bei diesem notwendig, für das erste Eingabesymbol eine Trennung durch Zustände 
vorzunehmen. In diesem Fall sind beide vom Startzustand erreichbaren Zustände nur durch 
unterschiedliche Eingabesymbole \(a \in A\) erreichbar. 

\subsubsection*{d)}
Diese Aussage ist nach Ausschlussprinzip korrekt. Darüber hinaus gilt per Definition, dass 
bei einem DEA für jeden Zustand Transitionen für alle Eingabesymbole definiert sind. Das führt 
zu der Erkenntnis, dass für jede durch einen DEA erkennbare Sprache ein oder mehrere 
"Auffangzustände" definiert werden können, welche dieser Eigenschaft entsprechen. 
\end{document}