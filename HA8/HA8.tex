\documentclass[12pt, a4paper]{article}

\usepackage[ngerman]{babel} 
\usepackage[T1]{fontenc}
\usepackage{amsfonts} 
\usepackage{setspace}
\usepackage{amsmath}
\usepackage{amssymb}
\usepackage{titling}
\usepackage{csquotes} % for \textquote{}
\usepackage{hyperref}
\usepackage{tikz}
\usepackage{stmaryrd}
\usetikzlibrary{arrows, automata, positioning}

\newcommand*{\qed}{\null\nobreak\hfill\ensuremath{\square}}
\newcommand*{\lqed}{\null\nobreak\hfill\ensuremath{\blacksquare}}
\newcommand*{\puffer}{\text{ }\text{ }\text{ }\text{ }}
\newcommand*{\gedanke}{\textbf{-- }}
\newcommand*{\gap}{\text{ }}
\newcommand*{\setDef}{\gap|\gap}
\newcommand*{\vor}{\textbf{Vor.:} \gap}
\newcommand*{\beh}{\textbf{Beh.:} \gap}
\newcommand*{\bew}{\textbf{Bew.:} \gap}
% Hab länger gebraucht um zu realisieren, dass das ne gute Idee wäre
\newcommand*{\R}{\mathbb R}

\newcommand{\HALT}{\text{HALT}_\text{TM}^\epsilon}
\newcommand{\STOPe}{\text{NONSTOP}_\text{TM}^\epsilon}
\newcommand{\STOP}{\text{NONSTOP}_\text{TM}}
\newcommand{\lb}{\llbracket}
\newcommand{\rb}{\rrbracket}

\newenvironment{noalign*}
 {\setlength{\abovedisplayskip}{0pt}\setlength{\belowdisplayskip}{0pt}%
  \csname flalign*\endcsname}
 {\csname endflalign*\endcsname\ignorespacesafterend}



\pagestyle{plain}
\allowdisplaybreaks

\setlength{\droptitle}{-11em}
\setlength{\jot}{12pt}
%\setlength{\hoffset}{-1in}     Wenn nötig
%\setlength{\textwidth}{535pt}  Wenn nötig

\title{Berechnungen und Logik\\Hausaufgabenserie 8}
\author{Henri Heyden, Nike Pulow \\ \small stu240825, stu239549}
\date{}


\begin{document}
\maketitle

\onehalfspacing
\vspace*{-2cm}
\subsection*{A1}
\vor \(L := \{\langle M \rangle \in \{0,1\}^* | \forall w : w \in L(M) \Leftrightarrow w = w^R\}\) \\
\beh \(L\) ist unentscheidbar. \\
\bew Sei \(f : \{0,1\}^* \rightarrow \{0,1\}^*, \langle M \rangle \mapsto \langle M \circ l \rangle\), wobei \(l \in L\) und \(\circ\) so, dass erst \(M\) berechnet wird und dann \(l\) berechnet wird.\\
\(M \circ l \in L\) gilt also genau dann, wenn \(M\) und \(l\) in einem akzeptierenden Zustand enden. \\
Dann gilt: \(w \in \HALT \Rightarrow f(w) \in L\) und \(w \not\in \HALT \Rightarrow f(w) \not\in L\).\footnote[1]{Wir haben hier die Äquivalenz aufgeteilt und die Zweite, also die \textquote{Rückrichtung} mittels Kontraposition gezeigt.} \\
Somit sind beide Richtungen gezeigt damit \(f\) Reduktionsfunktion für die Reduktion \(\HALT \le L\) ist. \\
Nach Satz \textquote{Eigenschaften der Reduktion} ist somit \(L\) nicht entscheidbar. \qed
\subsection*{A3}
\subsubsection*{a)}
Es gilt: \(\overline{\STOPe} = \HALT\). Wir zeigen also, dass \(\HALT\) erkennbar ist. \\
Folgende Turingmaschine erkennt \(\HALT\):\\
\(M : \{0,1\}^* \rightarrow \mathbb B, \langle w \rangle \mapsto \begin{cases}
    \text{wahr} & \text{hält } w \text{?} \\
    \text{falsch} & \text{sonst}
\end{cases}\) \\
Diese TM erkennt dann also alle haltenden Turingmaschinen, hält sie nicht interessiert uns das nicht, da wir nur zeigen wollten, dass \(\STOPe\) co-erkennbar ist. \qed \pagebreak
\subsubsection*{b)}
Definiere die TM \(H\) so, dass \(H\) für jede Eingabe hält, außer der leeren Eingabe, also \(\epsilon\). Sei \(f : \{0,1\}^* \rightarrow \{0,1\}^*, \langle w \rangle \mapsto \langle w \circ H \rangle\) so, dass \(w \circ H\) erst \(w\) simuliert und dann \(H\) simuliert für die gleiche Eingabe. \\
Dann gilt folgendes: \\
Ist \(\langle w\rangle \in \STOPe\), dann hält \(w \circ H\), also \(f(w)\) \textquote{decodiert} für keine Eingabe, also es gilt \(f(w) = \langle w \circ H \rangle \in \STOP\). \\
Ist \(\langle w\rangle \not\in \STOPe\), dann existiert ein Eingabewort, sodass \(w\) hält und damit dann auch \(w \circ H\), also gilt dann \(f(w) = \langle w \circ H \rangle \not\in \STOP\). \\
Somit eignet sich \(f\) als Reduktionsfunktion für die zu zeigende Reduktion \(\STOPe \le \STOP\). \qed

\subsection*{A4}
Es gilt \(\varphi_1, \varphi_3 \notin F_{AL}\) und \(\varphi_2, \varphi_4 \in F_{AL}\).\\
Eine für \(\varphi_2\) gültige Belegung \(\beta_1\) ist \(\llbracket \varphi_2 \rrbracket_{\beta_1} = 1\).\\
Eine für \(\varphi_4\) gültige Belegung \(\beta_2\) ist \(\llbracket \varphi_4 \rrbracket_{\beta_2} = 0\).

\subsection*{A5}
\subsubsection*{a)}
\begin{flalign*}
    \lb \varphi_1 \rb_{\beta}   & = (\lb \lnot X_0 \rb_{\beta} \wedge \lb Y_0 \rb_{\beta}) \vee ((\lb X_0 \rb_{\beta} \leftrightarrow \lb Y_0 \rb_{\beta}) \wedge \varphi_0) & \text{Definition Semantik}\\
                                & = (\lnot 0 \wedge 0) \vee ((0 \leftrightarrow 0) \wedge \top) & \beta, \varphi_0 = \top, \text{Basiselement} \\
                                & = (1 \wedge 0) \vee (1 \wedge \top) & \text{Auswertung Junktoren} \\
                                & = 0 \vee \top & \text{Auswertung } \vee \\
                                & = \top & \\
    \lb \varphi_2 \rb_{\beta}   & = (\lb \lnot X_1 \rb_{\beta} \wedge \lb Y_1 \rb_{\beta}) \vee (( \lb X_1 \rb_{\beta} \leftrightarrow \lb Y_1 \rb_{\beta}) \wedge \varphi_1) & \text{Definition Semantik} \\
                                & = (\lnot 0 \wedge 1) \vee ((0 \leftrightarrow 1) \wedge \top) & \text{Auswertung } \lnot, \leftrightarrow, \varphi_1=\top\\
                                & = (1 \wedge 1) \vee (0 \wedge \top) & \text{Auswertung } \wedge\\
                                & = 1 \vee 0 & \text{Auswertung } \vee\\
                                & = 1 = \top\\
    \lb \varphi_3 \rb_{\beta}   & = (\lb \lnot X_2 \rb_{\beta}) \wedge \lb Y_2 \rb_{\beta} \vee (( \lb X_2 \rb_{\beta} \leftrightarrow \lb Y_2 \rb_{\beta}) \wedge \varphi_2) & \text{Definition Semantik} \\
                                & = (\lnot 1 \wedge 0 ) \vee (( 1 \leftrightarrow 0) \wedge \top) & \text{Auswertung } \lnot, \leftrightarrow, \varphi_2 = \top\\
                                & = (0 \wedge 0) \vee ( 0 \wedge \top) & \text{Auswertung } \wedge \\
                                & = 0 \vee 0 & \text{Auswertung } \vee\\
                                & = 0 = \bot = \lb \varphi_3 \rb_{\beta} & \\
\end{flalign*}
\subsubsection*{b)}
Betrachte: \\
\textbf{(1)} \puffer \(\bigwedge_{i=0}^{n-1} (X_i \leftrightarrow Y_i)\)\\
\textbf{(2)} \puffer \(\bigvee_{i=0}^{n-1} ( \lnot X_i \wedge Y_1 \wedge\) \textbf{(3)})\\
\textbf{(3)} \puffer \(\bigvee_{j=i+1}^{n-1}(X_j \leftrightarrow Y_j))\)\\
\textbf{(1)} stellt sicher, dass die Bits von \(X\) und \(Y\) an der Stelle \(i\) gleich sind.
Kommt \textbf{(3)} zur Anwendung, dann sind \(X\) und \(Y\) an Stelle \(j\) immer gleich, während 
in \textbf{(2)} die große Disjunktion dafür sorgt, dass die Bits \(\lnot X_i\) und \(Y_i\) nicht beide zu 0 
ausgewertet werden und dass alle nachfolgenden Bits gleich sind.\\
Alle Teile der Relation sorgen also für einen Vergleich der beiden Binärzahlen \(X\) und \(Y\): Entweder 
alle Bits an Stelle \(i\) sind gleich \textbf{oder} es gibt eine Stelle \(i\), an der \(\lb X_1 \rb = 0\) und 
\(\lb Y_1 \rb = 1\) und an allen weiteren Positionen \(j\) sind die Bits gleich. \\
Zusammengefasst bedeutet dies: Entweder die durch \(X_n\) und \(Y_n\) repräsentierten Binärzahlen sind gleich, 
oder unterscheiden sich nur bis zu einer bestimmten Stelle \(i\) und alle nachfolgenden Bits sind gleich.\\

\subsection*{A6}
Die Menge aller aussagenlogischen Variablen, die in \(\varphi \in F_{AL}\) vorkommen, \(vars(\varphi) : 
F_{AL} \rightarrow \mathcal{P}(V_{AL})\), definieren wir induktiv wie folgt: \\
\textbf{IA:} \puffer Sei \(\varphi_0 \in F_{AL}\). Dann definieren wir \(vars(\varphi_0) := \{\varphi_0\}\).\\
\textbf{IS:} \puffer Sind \(\varphi_0, \dots, \varphi_n \in V_{AL}\) mit \(n \in \mathbb{N}_0\) durch 
beliebigen n-stelligen Junktor \(C\) verbunden, sodass \(C(\varphi_0, \dots, \varphi_n) \in F_{AL}\) gilt, 
dann definieren wir \(vars(C(\varphi_0, \dots, \varphi_n)) := \{\varphi_0, \dots, \varphi_n\}\).
\end{document}