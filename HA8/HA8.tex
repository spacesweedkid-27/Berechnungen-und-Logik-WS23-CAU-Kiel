\documentclass[12pt, a4paper]{article}

\usepackage[ngerman]{babel} 
\usepackage[T1]{fontenc}
\usepackage{amsfonts} 
\usepackage{setspace}
\usepackage{amsmath}
\usepackage{amssymb}
\usepackage{titling}
\usepackage{csquotes} % for \textquote{}
\usepackage{hyperref}
\usepackage{tikz}
\usetikzlibrary{arrows, automata, positioning}

\newcommand*{\qed}{\null\nobreak\hfill\ensuremath{\square}}
\newcommand*{\lqed}{\null\nobreak\hfill\ensuremath{\blacksquare}}
\newcommand*{\puffer}{\text{ }\text{ }\text{ }\text{ }}
\newcommand*{\gedanke}{\textbf{-- }}
\newcommand*{\gap}{\text{ }}
\newcommand*{\setDef}{\gap|\gap}
\newcommand*{\vor}{\textbf{Vor.:} \gap}
\newcommand*{\beh}{\textbf{Beh.:} \gap}
\newcommand*{\bew}{\textbf{Bew.:} \gap}
% Hab länger gebraucht um zu realisieren, dass das ne gute Idee wäre
\newcommand*{\R}{\mathbb R}

\newcommand{\HALT}{\text{HALT}_\text{TM}^\epsilon}
\newcommand{\STOPe}{\text{NONSTOP}_\text{TM}^\epsilon}
\newcommand{\STOP}{\text{NONSTOP}_\text{TM}}

\newenvironment{noalign*}
 {\setlength{\abovedisplayskip}{0pt}\setlength{\belowdisplayskip}{0pt}%
  \csname flalign*\endcsname}
 {\csname endflalign*\endcsname\ignorespacesafterend}



\pagestyle{plain}
\allowdisplaybreaks

\setlength{\droptitle}{-11em}
\setlength{\jot}{12pt}
%\setlength{\hoffset}{-1in}     Wenn nötig
%\setlength{\textwidth}{535pt}  Wenn nötig

\title{Berechnungen und Logik\\Hausaufgabenserie 8}
\author{Henri Heyden, Nike Pulow \\ \small stu240825, stu239549}
\date{}


\begin{document}
\maketitle

\onehalfspacing
\vspace*{-2cm}
\subsection*{A1}
\vor \(L := \{\langle M \rangle \in \{0,1\}^* | \forall w : w \in L(M) \Leftrightarrow w = w^R\}\) \\
\beh \(L\) ist unentscheidbar. \\
\bew Sei \(f : \{0,1\}^* \rightarrow \{0,1\}^*, \langle M \rangle \mapsto \langle M \circ l \rangle\), wobei \(l \in L\) und \(\circ\) so, dass erst \(M\) berechnet wird und dann \(l\) berechnet wird.\\
\(M \circ l \in L\) gilt also genau dann, wenn \(M\) und \(l\) in einem akzeptierenden Zustand enden. \\
Dann gilt: \(w \in \HALT \Rightarrow f(w) \in L\) und \(w \not\in \HALT \Rightarrow f(w) \not\in L\).\footnote[1]{Wir haben hier die Äquivalenz aufgeteilt und die Zweite, also die \textquote{Rückrichtung} mittels Kontraposition gezeigt.} \\
Somit sind beide Richtungen gezeigt damit \(f\) Reduktionsfunktion für die Reduktion \(\HALT \le L\) ist. \\
Nach Satz \textquote{Eigenschaften der Reduktion} ist somit \(L\) nicht entscheidbar. \qed
\subsection*{A3}
\subsubsection*{a)}
Es gilt: \(\overline{\STOPe} = \HALT\). Wir zeigen also, dass \(\HALT\) erkennbar ist. \\
Folgende Turingmaschine erkennt \(\HALT\):\\
\(M : \{0,1\}^* \rightarrow \mathbb B, \langle w \rangle \mapsto \begin{cases}
    \text{wahr} & \text{hält } w \text{?} \\
    \text{falsch} & \text{sonst}
\end{cases}\) \\
Diese TM erkennt dann also alle haltenden Turingmaschinen, hält sie nicht interessiert uns das nicht, da wir nur zeigen wollten, dass \(\STOPe\) co-erkennbar ist. \qed \pagebreak
\subsubsection*{b)}
Definiere die TM \(H\) so, dass \(H\) für jede Eingabe hält, außer der leeren Eingabe, also \(\epsilon\). Sei \(f : \{0,1\}^* \rightarrow \{0,1\}^*, \langle w \rangle \mapsto \langle w \circ H \rangle\) so, dass \(w \circ H\) erst \(w\) simuliert und dann \(H\) simuliert für die gleiche Eingabe. \\
Dann gilt folgendes: \\
Ist \(\langle w\rangle \in \STOPe\), dann hält \(w \circ H\), also \(f(w)\) \textquote{decodiert} für keine Eingabe, also es gilt \(f(w) = \langle w \circ H \rangle \in \STOP\). \\
Ist \(\langle w\rangle \not\in \STOPe\), dann existiert ein Eingabewort, sodass \(w\) hält und damit dann auch \(w \circ H\), also gilt dann \(f(w) = \langle w \circ H \rangle \not\in \STOP\). \\
Somit eignet sich \(f\) als Reduktionsfunktion für die zu zeigende Reduktion \(\STOPe \le \STOP\). \qed
\end{document}