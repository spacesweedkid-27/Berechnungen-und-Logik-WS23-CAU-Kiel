\documentclass[12pt, a4paper]{article}

\usepackage[ngerman]{babel} 
\usepackage[T1]{fontenc}
\usepackage{amsfonts} 
\usepackage{setspace}
\usepackage{amsmath}
\usepackage{amssymb}
\usepackage{titling}
\usepackage{csquotes} % for \textquote{}
\usepackage{hyperref}
\usepackage{tikz}
\usetikzlibrary{arrows, automata, positioning}

\newcommand*{\qed}{\null\nobreak\hfill\ensuremath{\square}}
\newcommand*{\lqed}{\null\nobreak\hfill\ensuremath{\blacksquare}}
\newcommand*{\puffer}{\text{ }\text{ }\text{ }\text{ }}
\newcommand*{\gedanke}{\textbf{-- }}
\newcommand*{\gap}{\text{ }}
\newcommand*{\setDef}{\gap|\gap}
\newcommand*{\vor}{\textbf{Vor.:} \gap}
\newcommand*{\beh}{\textbf{Beh.:} \gap}
\newcommand*{\bew}{\textbf{Bew.:} \gap}
% Hab länger gebraucht um zu realisieren, dass das ne gute Idee wäre
\newcommand*{\R}{\mathbb R}

\newcommand{\HALT}{\text{HALT}_\text{TM}^\epsilon}

\newenvironment{noalign*}
 {\setlength{\abovedisplayskip}{0pt}\setlength{\belowdisplayskip}{0pt}%
  \csname flalign*\endcsname}
 {\csname endflalign*\endcsname\ignorespacesafterend}



\pagestyle{plain}
\allowdisplaybreaks

\setlength{\droptitle}{-11em}
\setlength{\jot}{12pt}
%\setlength{\hoffset}{-1in}     Wenn nötig
%\setlength{\textwidth}{535pt}  Wenn nötig

\title{Berechnungen und Logik\\Hausaufgabenserie 8}
\author{Henri Heyden, Nike Pulow \\ \small stu240825, stu239549}
\date{}


\begin{document}
\maketitle

\doublespacing
\vspace*{-2cm}
\subsection*{A1}
\vor \(L := \{\langle M \rangle \in \{0,1\}^* | \forall w : w \in L(M) \Leftrightarrow w = w^R\}\) \\
\beh \(L\) ist unentscheidbar. \\
\bew Sei \(f : \{0,1\}^* \rightarrow \{0,1\}^*, M \mapsto M \circ l\), wobei \(l \in L\) und \(\circ\) so, dass erst \(M\) berechnet wird und dann \(l\) berechnet wird.\\
\(M \circ l \in L\) gilt also genau dann, wenn \(M\) und \(l\) in einem akzeptierenden Zustand enden. \\
Dann gilt: \(w \in \HALT \Rightarrow f(w) \in L\) und \(w \not\in \HALT \Rightarrow f(w) \not\in L\).\footnote[1]{Wir haben hier die Äquivalenz aufgeteilt und die Zweite, also die \textquote{Rückrichtung} mittels Kontraposition gezeigt.} \\
Somit sind beide Richtungen gezeigt damit \(f\) Reduktionsfunktion für die Reduktion \(\HALT \le L\) ist. \\
Nach Satz \textquote{Eigenschaften der Reduktion} ist somit \(L\) nicht entscheidbar. \qed
\end{document}