\documentclass[12pt, a4paper]{article}

\usepackage[ngerman]{babel} 
\usepackage[T1]{fontenc}
\usepackage{amsfonts} 
\usepackage{setspace}
\usepackage{amsmath}
\usepackage{amssymb}
\usepackage{titling}
\usepackage{csquotes} % for \textquote{}
\usepackage{hyperref}
\usepackage{tikz}
\usepackage{stmaryrd} % \llbracket and \rrbracket
\usetikzlibrary{arrows, automata, positioning}

\newcommand*{\qed}{\null\nobreak\hfill\ensuremath{\square}}
\newcommand*{\lqed}{\null\nobreak\hfill\ensuremath{\blacksquare}}
\newcommand*{\puffer}{\text{ }\text{ }\text{ }\text{ }}
\newcommand*{\gedanke}{\textbf{-- }}
\newcommand*{\gap}{\text{ }}
\newcommand*{\setDef}{\gap|\gap}
\newcommand*{\vor}{\textbf{Vor.:} \gap}
\newcommand*{\beh}{\textbf{Beh.:} \gap}
\newcommand*{\bew}{\textbf{Bew.:} \gap}
% Hab länger gebraucht um zu realisieren, dass das ne gute Idee wäre
\newcommand*{\R}{\mathbb R}
% Das hier wird sehr nett werden
\newcommand*{\vDashR}{\mathrel{\reflectbox{\(\vDash\)}}}
\newcommand*{\LeftrightvDash}{\vDash\vDashR}

\newenvironment{noalign*}
 {\setlength{\abovedisplayskip}{0pt}\setlength{\belowdisplayskip}{0pt}%
  \csname flalign*\endcsname}
 {\csname endflalign*\endcsname\ignorespacesafterend}



\pagestyle{plain}
\allowdisplaybreaks

\setlength{\droptitle}{-11em}
\setlength{\jot}{12pt}
%\setlength{\hoffset}{-1in}     Wenn nötig
%\setlength{\textwidth}{535pt}  Wenn nötig

\title{Berechnungen und Logik\\Hausaufgabenserie 9}
\author{Henri Heyden, Nike Pulow \\ \small stu240825, stu239549}
\date{}


\begin{document}
\maketitle

\doublespacing
\vspace*{-2cm}
\subsection*{A1}
\vor Sei \(\beta\) beliebige Belegung für die Formeln \(\varphi, \psi \in F_{AL}\). \\
\beh \(\neg(\varphi \vee \psi) \LeftrightvDash \neg\varphi \wedge \neg\psi\) \\
\bew\\
Fall 1.: \((\llbracket \varphi \rrbracket_\beta, \llbracket \psi \rrbracket_\beta) = (0,0)\). \\
Es gilt:
\begin{noalign*}
    & \llbracket \neg(\varphi \vee \psi) \rrbracket_\beta = f_\neg(\llbracket \varphi \vee \psi \rrbracket_\beta) = f_\neg(f_\vee(\llbracket \varphi \rrbracket_\beta, \llbracket \psi \rrbracket_\beta)) = f_\neg(f_\vee(0,0)) \\
    = & f_\neg(0) = 1 = f_\wedge(1,1) = f_\wedge(f_\neg(0), f_\neg(0)) = f_\wedge(f_\neg(\llbracket \varphi \rrbracket_\beta), f_\neg(\llbracket \psi \rrbracket_\beta)) \\
    = & f_\wedge(\llbracket \varphi \rrbracket_\beta, \llbracket \psi \rrbracket_\beta) = \llbracket\neg\varphi \wedge \neg\psi\rrbracket_\beta
\end{noalign*}
Andere Fälle analog oder mittels Tabelle. \qed \pagebreak
\subsection*{A2}
\vor \(n \in \mathbb N_0, \varphi_0, \dots \varphi_{n-1}\) Formeln. \\
\beh \(\neg\bigwedge_{i=0}^{n-1}\varphi_i \LeftrightvDash \bigvee_{i=0}^{n-1}\neg\varphi_i\) \\
\bew Wir zeigen mittels Induktion: \\
\textbf{(IB):} Es gilt: \(\neg(\wedge(\top)) \LeftrightvDash \neg(\top) \LeftrightvDash \bot \LeftrightvDash \vee(\bot) \LeftrightvDash \vee(\neg(\top))\)\footnote[1]{Hmm, ich wünschte, der Text wäre ein bisschen fetter\dots} \\
Anderer Fall analog. \\
\textbf{(IS):} Sei angenommen \textbf{(IH)} \(\neg\bigwedge_{i=0}^{n-2}\varphi_i \LeftrightvDash \bigvee_{i=0}^{n-2}\neg\varphi_i\).\\
Zu zeigen ist dann: \(\neg\bigwedge_{i=0}^{n-1}\varphi_i \LeftrightvDash \bigvee_{i=0}^{n-1}\neg\varphi_i\). \\
Es gilt: \begin{noalign*}
    \neg\bigwedge_{i=0}^{n-1}\varphi_i & \LeftrightvDash \neg\left(\bigwedge_{i=0}^{n-2}\varphi_i \wedge \varphi_{n-1}\right) & \text{| \textbf{(IB)} bzw. Bearbeitung von A1} \\
    & \LeftrightvDash \neg\bigwedge_{i=0}^{n-2}\varphi_i \vee \neg\varphi_{n-1} & \text{| \textbf{(IH)}, Ersetzungslemma} \\
    & \LeftrightvDash \bigvee_{i=0}^{n-2}\neg\varphi_i \vee \neg\varphi_{n-1} \LeftrightvDash \bigvee_{i=0}^{n-1}\neg\phi_i
\end{noalign*} \\
Somit sind Induktionsbasis und Induktionsschritt gezeigt. \qed
\end{document}