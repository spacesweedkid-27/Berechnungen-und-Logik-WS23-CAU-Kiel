\documentclass[12pt, a4paper]{article}

\usepackage[ngerman]{babel} 
\usepackage[T1]{fontenc}
\usepackage{amsfonts} 
\usepackage{setspace}
\usepackage{amsmath}
\usepackage{amssymb}
\usepackage{titling}
\usepackage{csquotes} % for \textquote{}
\usepackage{hyperref}
\usepackage{tikz}
\usepackage{stix}
\usepackage{stmaryrd} % \llbracket and \rrbracket
\usetikzlibrary{arrows, automata, positioning}

\newcommand*{\qed}{\null\nobreak\hfill\ensuremath{\square}}
\newcommand*{\lqed}{\null\nobreak\hfill\ensuremath{\blacksquare}}
\newcommand*{\puffer}{\text{ }\text{ }\text{ }\text{ }}
\newcommand*{\gedanke}{\textbf{-- }}
\newcommand*{\gap}{\text{ }}
\newcommand*{\setDef}{\gap|\gap}
\newcommand*{\vor}{\textbf{Vor.:} \gap}
\newcommand*{\beh}{\textbf{Beh.:} \gap}
\newcommand*{\bew}{\textbf{Bew.:} \gap}
% Hab länger gebraucht um zu realisieren, dass das ne gute Idee wäre
\newcommand*{\R}{\mathbb R}
% Das hier wird sehr nett werden
\newcommand*{\vDashR}{\mathrel{\reflectbox{\(\vDash\)}}}
\newcommand*{\LeftrightvDash}{\vDash\vDashR}
\newcommand{\lb}{\llbracket}
\newcommand{\rb}{\rrbracket}
\newcommand{\rbb}{\rrbracket_{\beta}}

\newenvironment{noalign*}
 {\setlength{\abovedisplayskip}{0pt}\setlength{\belowdisplayskip}{0pt}%
  \csname flalign*\endcsname}
 {\csname endflalign*\endcsname\ignorespacesafterend}



\pagestyle{plain}
\allowdisplaybreaks

\setlength{\droptitle}{-11em}
\setlength{\jot}{12pt}
%\setlength{\hoffset}{-1in}     Wenn nötig
%\setlength{\textwidth}{535pt}  Wenn nötig

\title{\scshape Berechnungen und Logik\\Hausaufgabenserie 9}
\author{\scshape Henri Heyden, Nike Pulow \\ \small stu240825, stu239549}
\date{}


\begin{document}
\maketitle

\doublespacing
\vspace*{-2cm}
\subsection*{A1}
\vor Sei \(\beta\) beliebige Belegung für die Formeln \(\varphi, \psi \in F_{AL}\). \\
\beh \(\neg(\varphi \vee \psi) \LeftrightvDash \neg\varphi \wedge \neg\psi\) \\
\bew\\
Fall 1.: \((\llbracket \varphi \rrbracket_\beta, \llbracket \psi \rrbracket_\beta) = (0,0)\). \\
Es gilt:
\begin{noalign*}
    & \llbracket \neg(\varphi \vee \psi) \rrbracket_\beta = f_\neg(\llbracket \varphi \vee \psi \rrbracket_\beta) = f_\neg(f_\vee(\llbracket \varphi \rrbracket_\beta, \llbracket \psi \rrbracket_\beta)) = f_\neg(f_\vee(0,0)) \\
    = & f_\neg(0) = 1 = f_\wedge(1,1) = f_\wedge(f_\neg(0), f_\neg(0)) = f_\wedge(f_\neg(\llbracket \varphi \rrbracket_\beta), f_\neg(\llbracket \psi \rrbracket_\beta)) \\
    = & f_\wedge(\llbracket \varphi \rrbracket_\beta, \llbracket \psi \rrbracket_\beta) = \llbracket\neg\varphi \wedge \neg\psi\rrbracket_\beta
\end{noalign*}
Fall \((\llbracket \varphi \rrbracket_\beta, \llbracket \psi \rrbracket_\beta) = (1,1)\) analog. \\
Fall \((\llbracket \varphi \rrbracket_\beta, \llbracket \psi \rrbracket_\beta) = (0,1)\). \\
Es gilt:
\begin{noalign*}
    & \llbracket \neg(\varphi \vee \psi) \rrbracket_\beta = f_\neg(\llbracket \varphi \vee \psi \rrbracket_\beta) = f_\neg(f_\vee(\llbracket \varphi \rrbracket_\beta, \llbracket \psi \rrbracket_\beta)) = f_\neg(f_\vee(0,1)) \\
    = & f_\neg(1) = 0 = f_\wedge(1,0) = f_\wedge(f_\neg(0), f_\neg(1)) = f_\wedge(f_\neg(\llbracket \varphi \rrbracket_\beta), f_\neg(\llbracket \psi \rrbracket_\beta)) \\
    = & f_\wedge(\llbracket \varphi \rrbracket_\beta, \llbracket \psi \rrbracket_\beta) = \llbracket\neg\varphi \wedge \neg\psi\rrbracket_\beta
\end{noalign*}
Fall \((\llbracket \varphi \rrbracket_\beta, \llbracket \psi \rrbracket_\beta) = (1,0)\) folgt aus Kommutativität \qed \\
Analog folgt der Beweis auch durch Ablesen einer Tabelle wo jeweilige Ausdrücke ausgewertet werden für alle möglichen Belegungen.\pagebreak
\subsection*{A2}
\vor \(n \in \mathbb N_0, \varphi_0, \dots \varphi_{n-1}\) Formeln. \\
\beh \(\neg\bigwedge_{i=0}^{n-1}\varphi_i \LeftrightvDash \bigvee_{i=0}^{n-1}\neg\varphi_i\) \\
\bew Wir zeigen mittels Induktion: \\
\textbf{(IB):} Es gilt: \(\neg(\wedge(\top)) \LeftrightvDash \neg(\top) \LeftrightvDash \bot \LeftrightvDash \vee(\bot) \LeftrightvDash \vee(\neg(\top))\) \\
Anderer Fall analog. \\
\textbf{(IS):} Sei angenommen \textbf{(IH)} \(\neg\bigwedge_{i=0}^{n-2}\varphi_i \LeftrightvDash \bigvee_{i=0}^{n-2}\neg\varphi_i\).\\
Zu zeigen ist dann: \(\neg\bigwedge_{i=0}^{n-1}\varphi_i \LeftrightvDash \bigvee_{i=0}^{n-1}\neg\varphi_i\). \\
Es gilt: \begin{noalign*}
    \neg\bigwedge_{i=0}^{n-1}\varphi_i & \LeftrightvDash \neg\left(\bigwedge_{i=0}^{n-2}\varphi_i \wedge \varphi_{n-1}\right) & \text{| \textbf{(IB)} bzw. Bearbeitung von A1} \\
    & \LeftrightvDash \neg\bigwedge_{i=0}^{n-2}\varphi_i \vee \neg\varphi_{n-1} & \text{| \textbf{(IH)}, Ersetzungslemma} \\
    & \LeftrightvDash \bigvee_{i=0}^{n-2}\neg\varphi_i \vee \neg\varphi_{n-1} \LeftrightvDash \bigvee_{i=0}^{n-1}\neg\phi_i
\end{noalign*} \\
Somit sind Induktionsbasis und Induktionsschritt gezeigt. \qed
\pagebreak
\subsection*{A4}
\vor \(\Phi \subseteq F_{AL}\) und \(\varphi \in F_{AL}\). \(\Phi \cup \{\lnot \varphi\}\) unerfüllbar. \\
\beh \(\Phi \vDash \varphi\) genau dann, wenn \(\Phi \cup \{\lnot \varphi\}\) unerfüllbar.\\
\bew Wenn \(\Phi \cup \{\lnot \varphi\}\) unerfüllbar ist, gilt \(\lb \Phi \cup \{\lnot \varphi\}\rb_{\beta}=0\) wegen der 
Definition von Erfüllbarkeit. Betrachte zwei Fälle:\\
\textbf{(1)} \(\lb \Phi \rb_{\beta} = 1\).\\
Damit \(\lb \Phi \cup \{\lnot \varphi\} \rb_{\beta} = 0\) gelten kann, muss \(\lb \lnot \varphi \rb_{\beta} = 1\) gelten. 
Daraus folgt \(\lb \varphi \rb_{\beta} = 1\). Dann gilt \(\Phi \vDash \varphi\), da \(\lb \Phi \rb_{\beta} = 1\), also 
\(\beta \vDash \Phi\), und \(\lb \varphi \rb_{\beta} = 1\), also \(\beta \vDash \varphi\). \\
\textbf{(2)} \(\lb \Phi \rbb = 0\)\\
Damit \(\lb \Phi \cup \{\lnot \varphi\} \rbb = 0\) gelten kann, muss \(\lb \varphi \rbb = 1\) gelten. Dann gilt auch 
\(\lb \varphi \rbb = 0\). \\
Dann wissen wir acuh, dass es keine passende Belegung \(\beta\) für \(\Phi\) und \(\varphi\) gibt, sodass \(\lb \Phi \rbb = 1\) 
und \(\lb \varphi \rbb = 1\) gelten. Es gilt also auf Grund der Nicht-Existenz von passenden Belegungen \(\beta\): \\
\(\Phi \vDash \varphi\).\\
\qed

\pagebreak

\subsection*{A6}
\subsubsection*{a)}
\begin{flalign*}
    & (\lnot X_0 \vee X_4) \wedge \lnot (X_1    \wedge (X_3 \vee \lnot X_2)) & \text{Distributivität}\\
    \LeftrightvDash \puffer & (\lnot X_0 \vee X_4)      \wedge \lnot ((X_1 \wedge X_3) \vee (X_1 \wedge \lnot X_2)) & \text{De Morgan}\\
    \LeftrightvDash \puffer & (\lnot X_0 \vee X_4)      \wedge \lnot (X_1 \wedge X_3) \wedge \lnot (X_1 \wedge \lnot X_2) & \text{De Morgan}\\
    \LeftrightvDash \puffer & (\lnot X_0 \vee X_4)      \wedge (\lnot X_1 \vee \lnot X_3) \wedge (\lnot X_1 \vee \lnot \lnot X_2) & \text{Doppelnegation} \\
    \LeftrightvDash \puffer & (\lnot X_0 \vee X_4)      \wedge (\lnot X_1 \vee \lnot X_3) \wedge (\lnot X_1 \vee X_2)
\end{flalign*}
Wir erhalten \(\varphi \bigwedge \bigvee = \{\{\lnot X_0, X_4\}, \{\lnot X_1, \lnot X_3\}, \{\lnot X_1, X_2\}\}\).
\subsubsection*{b)}
\begin{flalign*}
    & (\lnot X_0 \vee X_4) \wedge \lnot (X_1 \wedge (X_3 \vee \lnot X_2)) & \text{DeMorgan}\\
    \LeftrightvDash \puffer & (\lnot X_0 \vee X_4) \wedge \lnot X_1 \vee \lnot (X_3 \vee \lnot X_2) & \text{Kommutativität} \\
    \LeftrightvDash \puffer & \lnot X_1 \wedge (\lnot X_0 \vee X_4) \vee \lnot (X_3 \vee \lnot X_2) & \text{Distributivität} \\
    \LeftrightvDash \puffer & (\lnot X_1 \wedge \lnot X_0) \vee (\lnot X_1 \wedge X_4) \vee \lnot (X_3 \vee \lnot X_2) & \text{De Morgan}\\
    \LeftrightvDash \puffer & (\lnot X_1 \wedge \lnot X_0) \vee (\lnot X_1 \wedge X_4) \vee (\lnot X_3 \wedge \lnot \lnot X_2) & \text{Doppelnegation} \\
    \LeftrightvDash \puffer & (\lnot X_1 \wedge \lnot X_0) \vee (\lnot X_1 \wedge X_4) \vee (\lnot X_3 \wedge X_2)
\end{flalign*}
Wir erhalten \(\varphi \bigvee \bigwedge = \{\{\lnot X_1, \lnot X_0\}, \{\lnot X_1, X_4\}, \{\lnot X_3, X_2\}\}\).

\subsection*{A7}
Definiere \(\varphi := \bigwedge \bigvee \{\{X_2,X_1,X_5\}, \{\lnot X_4, X_2, \lnot X_3\}, \{\lnot X_1\}, \{X_4,X_5,\lnot X_2\},\)\\
\(\{\lnot X_4,X_1\}, \{X_2, \lnot X_5, \lnot X_3\}, \{X_3, X_1\}, \{\lnot X_5, \lnot X_2\}\}\)\\
Wir zeigen \(\varphi \vDash \bot\) mittels Resolutionsbeweis.
\begin{flalign*}
    \textbf{1.} \puffer     & \{\lnot X_1\}                 & \text{Voraussetzung}\\
    \textbf{2.} \puffer     & \{X_3, X_1\}                  & \text{Voraussetzung}\\
    \textbf{3.} \puffer     & \{X_3\}                       & \text{Resolution mit \(X_1\) aus 1 und 2}\\
    \textbf{4.} \puffer     & \{X_2, \lnot X_5, \lnot X_3\} & \text{Voraussetzung}\\
    \textbf{5.} \puffer     & \{X_2, \lnot X_5\}            & \text{Resolution mit \(X_3\) aus 3 und 4}\\
    \textbf{6.} \puffer     & \{\lnot X_5, \lnot X_2\}      & \text{Voraussetzung}\\
    \textbf{7.} \puffer     & \{\lnot X_5\}                 & \text{Resolution mit \(X_2\) aus 5 und 6}\\
    \textbf{8.} \puffer     & \{X_2, X_1, X_5\}             & \text{Voraussetzung}\\
    \textbf{9.} \puffer     & \{X_2, X_1\}                  & \text{Resolution mit \(X_5\) aus 7 und 8}\\
    \textbf{10.} \puffer    & \{X_2\}                       & \text{Resolution mit \(X_1\) aus 1 und 9}\\
    \textbf{11.} \puffer    & \{X_4, X_5, \lnot X_2\}       & \text{Voraussetzung}\\
    \textbf{12.} \puffer    & \{X_4, X_5\}                  & \text{Resolution mit \(X_2\) aus 10 und 11}\\
    \textbf{13.} \puffer    & \{X_4\}                       & \text{Resolution mit \(X_5\) aus 7 und 12}\\
    \textbf{14.} \puffer    & \{\lnot X_4, X_1\}            & \text{Voraussetzung}\\
    \textbf{15.} \puffer    & \{X_1\}                       & \text{Resolution mit \(X_4\) aus 13 und 14}\\
    \textbf{16.} \puffer    & \{\}                          & \text{Resolution mit \(X_1\) aus 1 und 15}
\end{flalign*}
Damit ist gezeigt, was zu zeigen war. \qed
\end{document}