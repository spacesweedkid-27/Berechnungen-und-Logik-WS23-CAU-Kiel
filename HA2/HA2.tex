\documentclass[12pt, a4paper]{article}

\usepackage[ngerman]{babel}
\usepackage[T1]{fontenc}
\usepackage{amsfonts}
\usepackage{setspace}
\usepackage{amsmath}
\usepackage{amssymb}
\usepackage{titling}


\newcommand*{\qed}{\null\nobreak\hfill\ensuremath{\square}}
\newcommand*{\puffer}{\text{ }\text{ }\text{ }\text{ }}
\newcommand*{\gedanke}{\textbf{-- }}
\newcommand*{\gap}{\text{ }}


\pagestyle{plain}
\allowdisplaybreaks

\setlength{\droptitle}{-11em}
\setlength{\jot}{12pt}

\title{Berechnungen und Logik\\Hausaufgabenserie 2}
\author{Henri Heyden, Nike Pulow \\ \small stu240825, stu239549}
\date{}


\begin{document}
\maketitle

\doublespacing

\subsection*{A1}

\subsubsection*{a)}

\begin{tabular}{l l || c c c c c}
    \( Q = \{ 0, 1, 2, 3 \} \) & \(A = \{a, b, c\}\)&\( \alpha, \delta \)&\( q \in Q \) & \( a \in A \) & \( \delta (q, a) \) & \( \alpha(q, a) \) \\ [0.5ex]
    \(B = \{ \epsilon, a, b, c \} \) & \(q_I = 0\) &&0 & a & 1 & a \\
    \( \iota = \epsilon\) & \(\omega = \epsilon\)&&0 & b & 2 & b \\
    &&&0 & c & 3 & c \\
    &&&1 & a & 1 & \(\epsilon\) \\
    &&&1 & b & 2 & b \\
    &&&1 & c & 3 & c \\
    &&&2 & a & 1 & a \\
    &&&2 & b & 2 & \( \epsilon \) \\
    &&&2 & c & 3 & c \\
    &&&3 & a & 1 & a \\
    &&&3 & b & 2 & b \\
    &&&3 & c & 3 & \( \epsilon \)\\[0.5ex]
\end{tabular}\\
\pagebreak
\subsubsection*{b)}
\begin{flalign*}
    f_M (abac) &= \epsilon \hat{\alpha} (0, abac) \omega ( \hat{\delta}(0,abac)) \\
    &= \epsilon \hat{\alpha} (0,abac) \epsilon \\
    &= \hat{\alpha} (0,abac) \\
    &= \hat{\alpha} (0,aba) \alpha (\hat{\delta}(0,aba), c) \\
    &= \hat{\alpha} (0,aba) \alpha (\delta (\hat{\delta}(0,ab),a), c) \\
    &= \hat{\alpha} (0,aba) \alpha (\delta (\delta(\hat{\delta}(0,a), b) ,a), c) \\
    &= \hat{\alpha} (0,aba) \alpha (\delta (\delta ( \delta(\hat{\delta}(0,\epsilon), a), b) ,a), c) \\
    &= \hat{\alpha} (0,aba) \alpha (\delta (\delta ( \delta(0 , a), b) ,a), c) \\
    &= \hat{\alpha} (0,aba) \alpha (\delta (\delta ( 1, b) ,a), c) \\
    &= \hat{\alpha} (0,aba) \alpha (\delta (2 ,a), c) \\
    &= \hat{\alpha} (0,aba) \alpha (1, c) \\
    &= \hat{\alpha} (0,aba) c \\
    &= \hat{\alpha} (0,ab) \alpha (\hat{\delta}(0, ab), a) c \\
    &= \hat{\alpha} (0,ab) \alpha (\delta (\hat{\delta}(0, a), b), a) c \\
    &= \hat{\alpha} (0,ab) \alpha (\delta (\delta (\hat{\delta}(0, \epsilon), a), b), a) c \\
    &= \hat{\alpha} (0,ab) \alpha (\delta (\delta (0, a), b), a) c \\
    &= \hat{\alpha} (0,ab) \alpha (\delta (1, b), a) c \\
    &= \hat{\alpha} (0,ab) \alpha (2, a) c \\
    &= \hat{\alpha} (0,ab) a c \\
    &= \hat{\alpha} (0,a) \alpha (\hat{\delta}(0, a), b) a c \\
    &= \hat{\alpha} (0,a) \alpha (\delta (\hat{\delta}(0, \epsilon), a), b) a c \\
    &= \hat{\alpha} (0,a) \alpha (\delta (0, a), b) a c \\
    &= \hat{\alpha} (0,a) \alpha (1, b) a c \\
    &= \hat{\alpha} (0,a) b a c \\
    &= \hat{\alpha} (0,\epsilon) \alpha (\hat{\delta}(0, \epsilon), a) b a c \\
    &= \hat{\alpha} (0,\epsilon) \alpha (0, a) b a c \\
    &= \hat{\alpha} (0,\epsilon) a b a c \\
    &= \epsilon a b a c \\
    &=  a b a c\\
    f_M (caa) \text{ } &= \epsilon \hat{\alpha} (0, abac) \omega ( \hat{\delta}(0,ca)a) \\
    & = \hat{\alpha}(0, ca) \alpha (\delta(\hat{\delta}(0,c)a)a) \\
    &= \hat{\alpha}(0, ca) \alpha (\delta(\delta(\hat{\delta}(0, \epsilon)c)a)a) \\
    &= \hat{\alpha}(0, ca) \alpha (\delta(\delta(0, c)a)a) \\
    &= \hat{\alpha}(0, ca) \alpha (\delta(3, a)a) \\
    &= \hat{\alpha}(0, ca) \alpha (1, a) \\
    &= \hat{\alpha}(0, ca) \epsilon \\
    &= \hat{\alpha}(0, c) \alpha (\hat{\delta}(0, c)a) \epsilon \\
    &= \hat{\alpha}(0, c) \alpha (3, a) \epsilon \\
    &= \hat{\alpha}(0, c) a \epsilon \\
    &= \hat{\alpha}(0, \epsilon) \alpha (\hat{\delta}(0, \epsilon)c)a \epsilon \\
    &= \hat{\alpha}(0, \epsilon) \alpha (0, c)a \epsilon \\
    &= \hat{\alpha}(0, \epsilon) c a \epsilon \\
    &= \epsilon c a \epsilon \\
    &= c a\\
\end{flalign*}

\subsubsection*{c)}
\( \{w \in A^* \mid f_M(w) = w\} = \{wx \in A^* \mid wx = wu^1 v^1 \wedge u,v \in A \wedge u \neq v\} \)

\subsection*{A2}
\begin{tabular}{l l || c c c | c c}
    \( Q = \{ \epsilon, 0, 1, 2, 3 \} \)& \(A = \{0,1,2,3\}\) & \(\alpha, \delta\)&\(q \in Q\)&\(a \in A\)&\(\delta(q,a)\)&\(\alpha(q,a)\)\\
    \( B = \{0,1,2,3\} \) & \(q_I = \epsilon \)&&\(\epsilon\)&0&0&0\\
    \(\iota = \epsilon\)&&&\(\epsilon\)&1&1&1\\
    \(\omega\):&&&\(\epsilon\)&2&2&2\\
    \(q \in Q\) & \(\omega (q) \)&&\(\epsilon\)&3&3&3\\
    0&0&&0&0&0&0\\
    1 & 3&&0&1&1&1\\
    2 & 2&&0&2&2&2\\
    3 & 1&&0&3&3&3\\
    &&&1&0&1&0\\
    &&&1&1&2&1\\
    &&&1&2&3&2\\
    &&&1&3&0&3\\
    &&&2&0&2&0\\
    &&&2&1&3&1\\
    &&&2&2&0&2\\
    &&&2&3&1&3\\
    &&&3&0&3&0\\
    &&&3&1&0&1\\
    &&&3&2&1&2\\
    &&&3&3&2&3\\
\end{tabular}\\\\
In dieser Modellierung einer sequentiellen Maschine haben die die Zustände so gewählt, dass diese immer die aktuelle Summe der Eingaben nach
den Voraussetzungen von \(f\) repräsentieren. Es ist die finale Ausgabefunktion \(\omega\) so gestaltet, dass in Abhängigkeit vom aktuellen
Zustand - sprich, der aktuellen Summe der vorangegangenen Eingabewerte - genau die Ausgabe getätigt wird, die benötigt wird, um die Summe
des Ausgabewortes auf 0 nach Voraussetzung zu bringen.\\
\subsection*{A3}
\begin{tabular}{l l || c c c | c c}
    \(Q = \{ \epsilon, !, 0, 1, 2, 3 \}\) & \(A = \{0,1,2,3 \}\) & \(\delta, \alpha\) & \(q\in Q\)& \(a \in A\) & \(\delta(q,a)\)&\(\alpha(q,a)\)\\
    \(B = \{ \epsilon, 0,1,2,3 \}\) & \(q_I = \epsilon\) && \(\epsilon\) & 0 & 0 & 3\\
    \(\iota = \epsilon\) &&& \(\epsilon\) & 1 & 1 & 2\\
    \(\omega\):&&& \(\epsilon\) & 2 & 2 & 1\\
    \(q \in Q \)&\(\omega (a)\)&& \(\epsilon\) & 3 & 3 & 0\\
    \( 0 \)&\(\epsilon\)&& 0& 0 & 0 & 3\\
    \( 1 \)& \(\epsilon\)&& 0 & 1 & 1 & 2\\
    \(2\)& \(\epsilon\) & & 0 & 2 & ! & 1\\
    3&\(\epsilon\)&&0&3&!&0\\
    &&&1&0&0&3\\
    &&&1&1&1&2\\
    &&&1&2&2&1\\
    &&&1&3&!&0\\
    &&&2&0&!&3\\
    &&&2&1&1&2\\
    &&&2&2&2&1\\
    &&&2&3&3&0\\
    &&&3&0&!&3\\
    &&&3&1&!&2\\
    &&&3&2&2&1\\
    &&&3&3&3&0\\
    &&&!&0&!&3\\
    &&&!&1&!&2\\
    &&&!&2&!&1\\
    &&&!&3&!&0\\
\end{tabular}\\\\
Es gilt \(f = f_M\), da in dieser sequentiellen Maschine der Zustand "!" der Art existiert, dass alle invaliden Eingaben - also solche, für die \(w \notin dom(f_M)\) gilt - durch diesen aufgefangen werden.
"!" ist durch \(\omega\) nicht abgedeckt, da die sequentielle Maschine in diesem Zustand nicht terminiert - Ausgaben für Wörter, die nicht in der Domain von \(f_M\) sind, sind also undefiniert.
Es ist denkbar, für Eingaben im Zustand ! das leere Wort als Ausgabe zu wählen, da, sobald dieser Zustand erreicht ist, die Maschine auf keinen Fall terminiert. 
Wir haben uns in diesem Fall für eine Fortsetzung der regulären Ausgaben entschieden.\\
Andere Voraussetzungen von \(f\) an \(f_M\) sind abgedeckt.
\end{document}