\documentclass[12pt, a4paper]{article}

\usepackage[ngerman]{babel} 
\usepackage[T1]{fontenc}
\usepackage{amsfonts} 
\usepackage{setspace}
\usepackage{amsmath}
\usepackage{amssymb}
\usepackage{titling}


\newcommand*{\qed}{\null\nobreak\hfill\ensuremath{\square}}
\newcommand*{\puffer}{\text{ }\text{ }\text{ }\text{ }}
\newcommand*{\gedanke}{\textbf{-- }}


\pagestyle{plain}
\allowdisplaybreaks

\setlength{\droptitle}{-11em}
\setlength{\jot}{12pt}

\title{Berechnungen und Logik\\Hausaufgabenserie 3}
\author{Henri Heyden, Nike Pulow \\ \small stu240825, stu239549}
\date{}


\begin{document}
\maketitle

\doublespacing
\subsection*{A1}
\subsubsection*{a)}
\(f_1\) erfüllt nicht das notwendige Kriterium des linear beschränktem Wachstums, denn es gilt folgendes: Für alle \(w \in A^*\) gilt: \(|f_1(w)| = |1^{|w|^2}| = |w|^2\).\\
Des Weiteren gilt jedoch: \(\forall c \in \mathbb N: \exists w \in A^*: |w|^2 > c\cdot(|x| + 1)\), mit \(w : |w| = 2c > 0\), denn es gilt dann: \(|w|^2 = 4c^2 > 2c^2 + c\) (Beachte \(|w| > 0\)). \qed
\subsubsection*{b)}
\(f_2\) erfüllt auch nicht das notwendige Kriterium des linear beschränktem Wachstums. 
Wir zeigen dafür \(\forall c \in \mathbb N: \exists w \in A^*: |f_2(w)| > c\cdot(|w| + 1)\) \\
Hierfür wähle \(w := v_0v_1v_3 \dots v_n\) mit \(n := \max(2c, 2)\) und \(v_i \ne \epsilon\) für \(i \in [n]\), dann gilt: 
\(|f_2(w)| = \frac{n^2 + n}{2} = \frac{4c^2 + 2c}{2} = 2c^2+c > 2c^2 + 2\) aufgrund der gaußschen Summenformel und der Definition von \(f_2\).
\\Dann gilt: \(|f_2(w)| > c\cdot(|w| + 1)\), was zu zeigen war. \qed
\subsection*{A2}

\subsection*{A3}
Da \(f\) und \(\lambda : B^* \rightarrow B^*, w \mapsto wv\) sequenziell\footnote[1]{\(\lambda\) ist tatsächlich sequenziell, betrachte dafür die jeweiligen Identitätsfunktionen als Zustands und Ausgabefunktionen und \(\phi : Q \rightarrow B^*, q \mapsto v\) als die finale Ausgabefunktion, für \(Q := \{z_0\}\).} sind, gilt nach Bonusaufgabe 7, dass \(\tilde{f}\) sequenziell ist, da \(\tilde{f} = \lambda \circ f\) gilt. \qed
\subsection*{A4}

\subsection*{A7}
maybe baby

\end{document}