\documentclass[12pt, a4paper]{article}

\usepackage[ngerman]{babel} 
\usepackage[T1]{fontenc}
\usepackage{amsfonts} 
\usepackage{setspace}
\usepackage{amsmath}
\usepackage{amssymb}
\usepackage{titling}


\newcommand*{\qed}{\null\nobreak\hfill\ensuremath{\square}}
\newcommand*{\puffer}{\text{ }\text{ }\text{ }\text{ }}
\newcommand*{\gedanke}{\textbf{-- }}


\pagestyle{plain}
\allowdisplaybreaks

\setlength{\droptitle}{-11em}
\setlength{\jot}{12pt}

\title{Berechnungen und Logik\\Hausaufgabenserie 3}
\author{Henri Heyden, Nike Pulow \\ \small stu240825, stu239549}
\date{}


\begin{document}
\maketitle

\doublespacing
\subsection*{A1}
\subsubsection*{a)}
\(f_1\) erfüllt nicht das notwendige Kriterium des linear beschränktem Wachstums, denn es gilt folgendes: Für alle \(w \in A^*\) gilt: \(|f_1(w)| = |1^{|w|^2}| = |w|^2\).\\
Des Weiteren gilt jedoch: \(\forall c \in \mathbb N: \exists w \in A^*: |w|^2 > c\cdot(|x| + 1)\), mit \(w : |w| = 2c > 0\), denn es gilt dann: \(|w|^2 = 4c^2 > 2c^2 + c\) (Beachte \(|w| > 0\)). \qed
\subsubsection*{b)}
\(f_2\) erfüllt auch nicht das notwendige Kriterium des linear beschränktem Wachstums. 
Wir zeigen dafür \(\forall c \in \mathbb N: \exists w \in A^*: |f_2(w)| > c\cdot(|w| + 1)\) \\
Hierfür wähle \(w := v_0v_1v_3 \dots v_n\) mit \(n := \max(2c, 3)\) und \(v_i \ne \epsilon\) für \(i \in [n]\), dann gilt: 
\(|f_2(w)| = \frac{n^2 + n}{2} = \frac{4c^2 + 2c}{2} = 2c^2+c > 2c^2 + 2\) aufgrund der gaußschen Summenformel und der Definition von \(f_2\).
\\Dann gilt: \(|f_2(w)| > c\cdot(|w| + 1)\), was zu zeigen war. \qed
\pagebreak
\subsection*{A2}
Zunächst zeigen wir \(f_M = f\) für \(x = 1^m\) per Induktion über \(m\).\\
Induktionsanfang: Für \(m = 0\) gilt: \(\hat{\alpha} (+1, 1^0) = \hat{\alpha}(+1, \epsilon = \epsilon)\) und für \(m = 1\) gilt:
\begin{flalign*}
    \hat{\alpha} (+1, 1^1) &= \hat{\alpha}(+1, \epsilon) \alpha (\hat{\delta}(+1, \epsilon), 1) & \text{Eigenschaft \(\hat{\alpha}\)}\\
    &= \epsilon \alpha(+1, 1) &\\
    &=\epsilon 0 = 0
\end{flalign*}
Induktionsschritt: Für \(m > 1\) gilt mit \(m = m_0 + (m-1), m_0 =1\):
\begin{flalign*}
    \hat{\alpha}(+1, 1^{m_0+(m-1)}) &= \hat{\alpha}(+1, 1^{m_0})\alpha(\hat{\delta}(+1, 1^{m_0}), 1^{m-1}) & \text{Eigenschaft \(\hat{\alpha}\)}\\
    &= \hat{\alpha}(+1, 1^{m_0}) \alpha(+1, 1^{m-1}) & \text{\(\hat{\delta}\) bekannt}\\
    &= \hat{\alpha}(+1, 1^1)\alpha (+1, 1^{m-1}) & \text{\(\hat{\alpha}(+1,1)\) bekannt}\\
    &= 0 \alpha(+1,1^{m-1}) \\
    &= 0^m
\end{flalign*}
Nun zeigen wir \(f_M = f\) für den Fall \(x = 1^m 0y\), hier per Induktion über \(y\).\\
Induktionsanfang: Für \(\vert y \vert = 0\), also \(y = \epsilon\) gilt:
\begin{flalign*}
    \hat{\alpha}(+1, 1^m0) &= \hat{\alpha}(+1, 1^m)\alpha(\hat{\delta}(+1, 1^m), 0) & \text{Eigenschaft \(\hat{\alpha}\)}\\
    &= 0^m \alpha(\hat{\delta}(+1, 1^m), 0) & \text{\(\hat{\alpha}(+1,1^m)\) bekannt}\\
    &= 0^m \alpha(+1,0) & \text{\(\hat{\delta}\) bekannt}\\
    &= 0^m 1\\
    &= 0^m1y = f(x) & y = \epsilon
\end{flalign*}
Induktionsschritt: Für \(\vert y \vert > 0\) gilt:
\begin{flalign*}
    \hat{\alpha}(+1, 1^m0y) &= \hat{\alpha}(+1, 1^m0) \alpha(\hat{\delta}(+1,1^m0), y) &\\
    &= \hat{\alpha}(+1, 1^m0)\alpha(\delta(\hat{\delta}(+1, 1^m),0),y)&\\
    &= \hat{\alpha}(+1, 1^m0)\alpha(\delta(+1, 0), y) & \text{\(\hat{\delta}\) bekannt}\\
    &= \hat{\alpha}(+1, 1^m0)\alpha(=, y) &\\
    &= \hat{\alpha}(+1, 1^m)\alpha(\hat{\delta}(+1, 1^m), 0) \alpha(=,y) & \text{Eigenschaft } \hat{\alpha}\\
    &= \hat{\alpha}(+1, 1^m)\alpha(+1,0) \alpha(=,y) & \hat{\delta} \text{ bekannt}\\
    &= \hat{\alpha}(+1, 1^m)1 \alpha(=,y) & \alpha(+1,0) \text{ bekannt}\\
    &= \hat{\alpha}(+1, 1^m)1y & \alpha(=,y) = y\\
    &=0^m1y &\hat{\alpha}(+1, 1^m) \text{ bekannt}
\end{flalign*}
Damit ist gezeigt, was zu zeigen war. \qed
\pagebreak
\subsection*{A3}
Da \(f\) und \(\lambda : B^* \rightarrow B^*, w \mapsto wv\) sequenziell\footnote[1]{\(\lambda\) ist tatsächlich sequenziell, betrachte dafür die jeweiligen Identitätsfunktionen als Zustands und Ausgabefunktionen und \(\phi : Q \rightarrow B^*, q \mapsto v\) als die finale Ausgabefunktion, für \(Q := \{z_0\}\).} sind, gilt nach Bonusaufgabe 7, dass \(\tilde{f}\) sequenziell ist, da \(\tilde{f} = \lambda \circ f\) gilt. \qed
\subsection*{A4}
\subsubsection*{a)}
Induktionsbasis: Sei \(u,v \in A^*, |u| = |v| = 1\). Dann gilt:
\begin{flalign*}
    & \hat\delta(q, uv) & \\
    = & \delta(\hat\delta(q, u), v) & \text{| Def. \(\hat\delta\)} \\
    = & \delta(\delta(\hat\delta(q, \epsilon), u), v) &  \\
    = & \delta(\delta(q, u), v) & \text{| \(2 \cdot\) Def. \(\hat\delta\)} \\
    = & \delta(\delta(\hat\delta(\hat\delta(q, \epsilon), \epsilon), u), v) &  \\
    = & \delta(\hat\delta(\hat\delta(q, \epsilon), u), v) & \text{| Def. \(\hat\delta\)}  \\
    = & \delta(\hat\delta(\hat\delta(\hat\delta(q,\epsilon), u), \epsilon), v) & \\
    = & \hat\delta(\hat\delta(\hat\delta(q, \epsilon), u), v) & \\
    = & \hat\delta(\hat\delta(q, u), v)
\end{flalign*}
Hier sieht man auch, dass die Basis gilt, wenn eines der Wörter das leere Wort ist, da wir das im Beweis selber gezeigt hatten zwischen Schritt 3 und 6. \\
Induktionsschritt: Sei \(w \in A^*, a \in A\). Dann gilt:
\begin{flalign*}
    & \hat\delta(q, wa) & \\
    = & \delta(\hat\delta(q, w), a) & \text{| Def. \(\hat\delta\)}  \\
    = & \delta(\hat\delta(\hat\delta(q, w), \epsilon), a) & \\
    = & \hat\delta(\hat\delta(q, w), a)
\end{flalign*}
Damit wurde die Aussage induktiv gezeigt. \qed
\subsubsection*{b)}
\pagebreak
\subsection*{A7}
\textbf{Voraussetzung:} \(A,B,C\) sind Mengen und \(f: B \rightharpoonup C\) und \(g: A \rightharpoonup B\) partielle Funktionen. \(f \circ g\) 
ist eine Komposition, gegeben durch:\\
\begin{equation*}
    f \circ g : A \rightharpoonup C, x \mapsto (f \circ g)(x) :=
    \begin{cases}
    f(g(x)), & \text{falls } g(x) \text{und } f(g(x)) \text{ definiert,} \\
    \text{undefiniert}, & \text{sonst}
    \end{cases}
\end{equation*}
\textbf{Behauptung:} Sind \(f\) und \(g\) sequentielle Funktionen, so ist auch ihre Komposition eine sequentielle Funktion.\\
\textbf{Beweis:} \\
Wir zeigen, dass die Komposition \(f \circ g\) linear beschränktes Abstandswachstum besitzt. Zu zeigen ist also:
\(\exists c \in \mathbb{N}\) mit \(d(f \circ g(x), f \circ g(y)) \leq c \cdot d(x,y)\).\\
Wenn \(f\) und \(g\) sequentielle Funktionen sind, dann gilt:\\
(1) \(d(g(x_1),g(y_1)) \leq c_1 \cdot d(x_1, y_1)\) für alle \(x_1,y_1 \in A\) und\\
(2) \(d(f(x_2),f(y_2)) \leq c_2 \cdot d(x_2, y_2)\) für alle \(x_2,y_2 \in B\).\\
Durch die Definition der Komposition wissen wir, dass \((f \circ g)(x) = f(g(x))\) gilt, falls \(g(x)\) und \(f(g(x))\) definiert sind.\\
Bekannt durch die Voraussetzung ist, wenn ein \(x \in dom(g)\) eine valide Eingabe ist, dann bildet \(g\) auf \(B\) ab, wobei nach 
Voraussetzung \(B = dom(f)\) gilt. Für jede Eingabe \(x \in dom(g)\) ist also auch \(f(g(x))\) definiert.\\
Wähle nun ein \(c \in \mathbb{N}\) so, dass \(c := max(c_1, c_2)\) so gilt, dass:\\
\(d(f(g(x)), f(g(y))) \leq c \cdot d(x,y)\) für alle \(x,y \in dom(g)\).\\
Damit ist gezeigt, was zu zeigen war. \qed

\end{document}